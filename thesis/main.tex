\documentclass[
10pt, % Main document font size
a4paper, % Paper type, use 'letterpaper' for US Letter paper
oneside, % One page layout (no page indentation)
%twoside, % Two page layout (page indentation for binding and different headers)
headinclude,footinclude, % Extra spacing for the header and footer
BCOR5mm, % Binding correction
]{scrartcl}

\usepackage{listings}
\usepackage{dirtree}
\usepackage{adjustbox}
\usepackage{hyperref}
\usepackage{fontawesome5}

\lstset{
    backgroundcolor=\color{white},
    language=bash,
    basicstyle=\ttfamily\small, 
    keywordstyle=[1]{\color{blue}}, 
    morekeywords=[1]{pki, ipsec, ip, xfrm, conn, EAP, RSA, ECDSA, openssl, qemu, kvm, chmod, perf, pmap}, 
    deletekeywords=[1]{type,in},
    keywordstyle=[2]{\color{green!40}},
    morekeywords=[2]{\$,}, 
    keywordstyle=[3]{\color{red}},
    morekeywords=[3]{leftsourceip, right, rightsubnet, auto, ike, leftauth, eap_identity, rightauth,leftcert,also,keyexchange,left,leftsubnet,forceencaps,compress,type,fragmentation,rekey,rightid,rightsourceip,rightdns,leftsendcert,src,dst,proto,auth,trunc, enc, anti,replay,window }, 
    deletekeywords=[3]{ipsec-ike},
    keywordstyle=[4]{\color{gray}},
    morekeywords=[4]{Certificate, Data, Version, Serial Number, Signature, Algorithm, Issuer, Validity, Subject, Public, Key, Info, Modulus, Exponent, Signature, Not, Before, After, Serial, Number, RSA, pub, X509v3, extensions, NIST, CURVE, ASN1, OID, Authority, Identifier, Alternative, Name, Extended, Usage },
    stringstyle=\color{red},
    breaklines=true,
    frame=single, 
    framesep=7pt,
}
%%%%%%%%%%%%%%%%%%%%%%%%%%%%%%%%%%%%%%%%%
% Arsclassica Article
% Structure Specification File
%
% This file has been downloaded from:
% http://www.LaTeXTemplates.com
%
% Original author:
% Lorenzo Pantieri (http://www.lorenzopantieri.net) with extensive modifications by:
% Vel (vel@latextemplates.com)
%
% License:
% CC BY-NC-SA 3.0 (http://creativecommons.org/licenses/by-nc-sa/3.0/)
%
%%%%%%%%%%%%%%%%%%%%%%%%%%%%%%%%%%%%%%%%%

%----------------------------------------------------------------------------------------
%	REQUIRED PACKAGES
%----------------------------------------------------------------------------------------

\usepackage[
nochapters, % Turn off chapters since this is an article        
beramono, % Use the Bera Mono font for monospaced text (\texttt)
eulermath,% Use the Euler font for mathematics
pdfspacing, % Makes use of pdftex’ letter spacing capabilities via the microtype package
dottedtoc % Dotted lines leading to the page numbers in the table of contents
]{classicthesis} % The layout is based on the Classic Thesis style

\usepackage{arsclassica} % Modifies the Classic Thesis package

\usepackage[T1]{fontenc} % Use 8-bit encoding that has 256 glyphs

\usepackage[utf8]{inputenc} % Required for including letters with accents

\usepackage{graphicx} % Required for including images
\graphicspath{{Figures/}} % Set the default folder for images

\usepackage{enumitem} % Required for manipulating the whitespace between and within lists

\usepackage{lipsum} % Used for inserting dummy 'Lorem ipsum' text into the template

\usepackage{subfig} % Required for creating figures with multiple parts (subfigures)

\usepackage{amsmath,amssymb,amsthm} % For including math equations, theorems, symbols, etc

\usepackage{varioref} % More descriptive referencing

%----------------------------------------------------------------------------------------
%	THEOREM STYLES
%---------------------------------------------------------------------------------------

\theoremstyle{definition} % Define theorem styles here based on the definition style (used for definitions and examples)
\newtheorem{definition}{Definition}

\theoremstyle{plain} % Define theorem styles here based on the plain style (used for theorems, lemmas, propositions)
\newtheorem{theorem}{Theorem}

\theoremstyle{remark} % Define theorem styles here based on the remark style (used for remarks and notes)

%----------------------------------------------------------------------------------------
%	HYPERLINKS
%---------------------------------------------------------------------------------------

\hypersetup{
%draft, % Uncomment to remove all links (useful for printing in black and white)
colorlinks=true, breaklinks=true, bookmarks=true,bookmarksnumbered,
urlcolor=webbrown, linkcolor=RoyalBlue, citecolor=webgreen, % Link colors
pdftitle={}, % PDF title
pdfauthor={\textcopyright}, % PDF Author
pdfsubject={}, % PDF Subject
pdfkeywords={}, % PDF Keywords
pdfcreator={pdfLaTeX}, % PDF Creator
pdfproducer={LaTeX with hyperref and ClassicThesis} % PDF producer
} 
\hyphenation{Fortran hy-phen-ation} 

%----------------------------------------------------------------------------------------
%	TITLE AND AUTHOR(S)
%----------------------------------------------------------------------------------------
\title{\normalfont\spacedallcaps{IKEv2 Testing}} 
%\subtitle{Subtitle}
\author{\spacedlowsmallcaps{Davide De Zuane \& Rahmi El Mechri}} 
\date{}

%----------------------------------------------------------------------------------------
%	DOCUMENT BODY
%----------------------------------------------------------------------------------------

\begin{document}

%----------------------------------------------------------------------------------------
%	HEADERS
%----------------------------------------------------------------------------------------
\renewcommand{\sectionmark}[1]{\markright{\spacedlowsmallcaps{#1}}}
\lehead{\mbox{\llap{\small\thepage\kern1em\color{halfgray} \vline}\color{halfgray}\hspace{0.5em}\rightmark\hfil}} 
\pagestyle{scrheadings} 

%----------------------------------------------------------------------------------------
%	TABLE OF CONTENTS & LISTS OF FIGURES AND TABLES
%----------------------------------------------------------------------------------------
\maketitle
\newpage
\setcounter{tocdepth}{2} 
\tableofcontents 

%----------------------------------------------------------------------------------------
%	ABSTRACT
%----------------------------------------------------------------------------------------
\section*{Abstract} 

%----------------------------------------------------------------------------------------
%	INTRODUCTION
%----------------------------------------------------------------------------------------

\newpage
\section{Introduction}
Lo scopo di questa tesina è stimare le risorse necessarie per l'utilizzo di IPsec, una suite di protocolli al livello network per realizzare comunicazioni sicure.
IPsec si basa sull'instaurazione di Security Associatons (SA), definite dal protocollo ISAKMP, le quali definiscono uno stato condiviso tra i due host della comunicaizone per utilizzare
primitive crittografiche. La parte critica di questo processo è lo scambio di chiavi, il quale avviene attraverso il protocollo IKE (Internet Key Exchange definito nell'RFC\cite{rfc7296}).
L'implementazione presa in considerazione è Strongswan poichè la più completa e diffusa. 
Questo lavoro si pone come parte di un progetto di sperimentazione più ampio che ha come obiettivo l'applicazione protocolli utilizzati già sull'internet alle comunicaizoni satellitari.
\\

\noindent
Infatti, l'utilizzo di satelliti Geo Stazionari richiede un'attenta analisi delle trasmissioni, dato che introducono elevati tempi di viaggio e canali molto rumorosi.
L'obiettivo è determianre se l'impiego di Strongswan è adatto a questa applicazione, ed esaminare delle possibili configurazione da utilizzare.
\\

\noindent
Procederemo con una descrizione su come abbiamo deciso di eseguire il testing e tutte le configurazioni che abbiamo eseguito.
Seguirà poi la descrizione delle configurazioni prese in esame.
%----------------------------------------------------------------------------------------
%	CONFIGURATION SETUP
%----------------------------------------------------------------------------------------
\newpage
\section{Setup}

Andiamo a vedere nel dettaglio l'ambiene e la configurazione che abbiamo utilizzato per realizzare i test. Per verificare le capacità di IKE abbiamo previsto:
\begin{itemize}
    \item $3$ modalità di autenticazione;
    \item $2$ chiper suite differenti da utlizzare.
\end{itemize}

\noindent
Nella fase di sperimentazione abbiamo utilizzato le seguenti convenzioni:
\begin{itemize}
    \item \textbf{Initiator}: l'host che invia la richiesta di stabilire una SA;
    \item \textbf{Responder}: l'host che risponde alle richieste.
\end{itemize}


\subsection{Environment}

Per simulare i due host della comunicazione abbiamo creato due macchine virtuali tramite l'utilizzo di qemu/kvm, questo per avere 
delle performance il più possibile simili a quelle reali.

\noindent
Le macchine virtuali utilizzato hanno le seguenti specifiche:

\begin{itemize}
    \item \textit{Processore}: 2 core (flag \lstinline|-smp|)
    \item \textit{Memoria}: 2048MB (flag \lstinline|-m|)
    \item \textit{OS}: Debian 11
    \item \textit{Network}: Bridge
\end{itemize}

\noindent
Le macchine virtuali sono state create utilizzando \lstinline|qemu/kvm| tramite i seguenti comandi è possibile creare la macchina virtuale.
\newline\newline\noindent
Per prima cosa è necessario creare un disco immagine.
\begin{lstlisting}
$ qemu-img create -f qcow2 disk.img 10G 
\end{lstlisting}
\vspace*{0.2cm}
Ora avviamo la macchina virtuale utilizzando il seguente comando.
\begin{lstlisting}
$ qemu-system-x86_64 -smp 2 -m 2G -hda disk.img -cdrom <debian_iso> \
    -net bridge,br=virbr0 -enable-kvm & disown
\end{lstlisting}

\vspace*{0.5cm}
\noindent
Un procedimento simile si applica per l'altra macchina virtuale. Se non si vuole proseguire in questo modo 
si può utilizzare l'interfaccia grafica fornita da \lstinline|virt-manager|.



\subsection{Configuration} 

I file e le directory coinvolte nel processo di configurazione sono i seguenti. Dato che una delle principali modifiche di IKEv2 
rispetto alla versione precedente è la possibilità di autenticazione tramite certificati andremo ad agire sulle directory che li contengono.
\\

    \dirtree{%
        .1 /etc.
        .2 ipsec.conf.
        .2 ipsec.secrets.
        .2 ipsec.d.
        .3 cacerts.
        .3 certs.
        .3 private.
    }
    
\begin{itemize}
    \item Il file \lstinline|ipsec.conf|\footnotemark[1] specifica la maggior parte delle configurazioni e le informazioni di controllo per il sottosistema IPsec (ulteriori specifiche e sintassi sono disponibili al seguente \href{https://linux.die.net/man/5/ipsec.conf}{link}).
    \item Il file \lstinline|ipsec.secrets|\footnotemark[1] continene i segreti che poi verrranno utilizzati nella fase di autenticazione (ulteriori specifiche al seguenti \href{https://linux.die.net/man/5/ipsec.secrets}{link}).
\end{itemize}

\footnotetext[1]{Le configurazioni utilizzate si trovano in \hyperlink{configuration}{appendice}. }

\subsubsection*{Certificati}

Una delle principali novità che introduce \lstinline|IKEv2| è la possibilità di eseguire l'autenticazione tra certificati X.509. In fase di testing abbiamo preso in considerazione due tipi di certificati:

\begin{itemize}
    \item Certificati RSA
    \item Certificati ECDSA
\end{itemize}

\noindent
A partire da una chiave pubblica è necessario realizzare un certificato di chiave pubblica e questo richiede la chiave privata di una CA. 
Nel nostro caso ci siamo creati dei certificati da CA e li abbiamo ditribuiti manualmente tra i due host.
Per la generazione abbiamo utilizzato il tool \lstinline|pki|.

\subsubsection*{CA Certificate}
Partiamo con la generazione dei certificati da Certification Authority, di seguito sono riportati i due comandi da utilizzare. Ne occorrono due poichè per firmare
i certificati ECDSA occorre una chiave con lo stesso schema.
\vspace*{0.2cm}
\begin{lstlisting}
$ pki --gen --type rsa --size 2048 --outform pem > 'ca.rsa.key.pem'
$ pki --gen --type ecdsa --size 256 --outform pem > 'ca.ecdsa.key.pem'
\end{lstlisting} 

\vspace*{0.2cm}
\noindent
Ora utilizziamo la chiave privata per firmare il certificato di chiave pubblica.
\begin{lstlisting}
$ pki --self --ca --lifetime 3650 --in 'ca.<type>.key.pem' --type <type> \
     --dn "CN=CA" --outform pem > ca.<type>.cert.pem
\end{lstlisting}

\noindent
Occorre poi distribuire questi due certificati ai due host, vanno messi all'intenro della directory \lstinline|cacerts|.

\subsubsection*{Host Certificate}
Passiamo ora a generare i certificati che gli host andranno ad utilizzare nella fase di autenticazione, occorre generare la coppia chiave privata, chiave pubblica.
\vspace*{0.2cm}
\begin{lstlisting}
$ pki --gen --type ecdsa --size 256 --outform pem > 'host.ecdsa.key'
$ pki --gen --type rsa --size 2048 --outform pem > 'host.rsa.key'
\end{lstlisting} 

\vspace*{0.2cm}
\noindent
E' buona norma salvare le chiavi all'interno della directory \lstinline|private|. Ora andiamo ad estrarre la chiave pubblica 
da quella appena genrata e la firmiamo con la chiave delle CA del passo precedente.
\begin{lstlisting}
$ pki --pub --in 'host.rsa.key' --type rsa | pki --issue --lifetime 1825 \
    --cacert 'ca.rsa.cert.pem' --cakey 'ca.rsa.key.pem'                  \
    --dn "CN=<Host_IP>" --san @<Host_IP> --san <Host_IP>                 \
    -- flag serverAuth --outform pem > 'host.rsa.cert.pem'
\end{lstlisting}

\noindent
Si procede in maniera analoga con le opportune modifiche anche per il certificato \lstinline|ECDSA|. Questi vanno poi posizonati all'intenro della
directory \lstinline|certs|.

\subsubsection{Mschap}
Il riassunto della configurazione è mostrato in tabella, per l'initiator e il responder sono riportate le modalità della loro autenticazione.
\begin{center}
    \setlength{\arrayrulewidth}{0.4mm}
    \renewcommand{\arraystretch}{1.3}
    \begin{tabular}{|l|l|}
        \hline
        \multicolumn{2}{|c|}{\textbf{Configuration}} \\
        \hline
        \textit{Initiator} & EAP-Mschapv2 \\
        \textit{Responder} & RSA Certificate $2048$ \\
        \textit{Chiper Suite} & $AES\_CBC\_128\_HMAC\_SHA2\_256\_128\_DH\_ECP\_256$ \\
        \hline
    \end{tabular}
\end{center}

\noindent
Esaminando gli scambi di \lstinline|IKE_AUTH| osserviamo che questa modalità richiede in totale $4$ exchange. 

\subsubsection{RSA}

\begin{center}
    \setlength{\arrayrulewidth}{0.4mm}
    \renewcommand{\arraystretch}{1.3}
    \begin{tabular}{|l|l|}
        \hline
        \multicolumn{2}{|c|}{\textbf{Configuration}} \\
        \hline
        \textit{Initiator} & RSA Certificate $2048$ \\
        \textit{Responder} & RSA Certificate $2048$ \\
        \textit{Chiper Suite} & $AES\_CBC\_128\_HMAC\_SHA2\_256\_128\_DH\_ECP\_256$ \\
        \hline
    \end{tabular}
\end{center}
\vspace*{0.2cm}
\noindent
Utilizzando certificati \lstinline|RSA| si osserva che la dimensione di un certificato eccede la dimensione massima di un pacchetto IP
per tali motivi si ha la frammentazione: in cui il contenuto del pacchetto, poichè eccede la dimenisone massima del campo \textit{data}, viene
spezzato in più paccetti.
\newline  \\
\noindent
Anando ad esaminare il certificato, si osserva che ha una dimensione pari a $1032$ byte, di cui abbiamo:
\begin{itemize}
    \item $256$ byte per la rappresentazoine del modulo;
    \item $1$ byte per la rappresentazione dell'esponente di cifratura
    \item $384$ byte per la firma 
    \item i restanti byte sono esaminati in  \hyperlink{certificati}{appendice}.
\end{itemize} 

\noindent
Idealmente durante la fase \lstinline|IKE_AUTH| dovrebbe essere presente un solo scambio, in cui i due si scambiano reciprocamente i certificati. Tuttavia,
data la dimensioni di quest'ultimi, gli scambi effettivi risultano essere in totale $2$. 

\subsubsection{ECDSA}

\begin{center}
    \setlength{\arrayrulewidth}{0.4mm}
    \renewcommand{\arraystretch}{1.3}
    \begin{tabular}{|l|l|}
        \hline
        \multicolumn{2}{|c|}{\textbf{Configuration}} \\
        \hline
        \textit{Initiator} & ECDSA Certificate 256\\
        \textit{Responder} & ECDSA Certificate 256 \\
        \textit{Chiper Suite} & $AES\_CBC\_128\_HMAC\_SHA2\_256\_128\_DH\_ECP\_256$ \\
        \hline
    \end{tabular}
\end{center}
\vspace*{0.2cm}
\noindent
Si osserva che i certificati \lstinline|ECDSA| hanno una dimensione ridotta rispetto a quella dei certifcati \lstinline|RSA|, infatti quello utilizzato nel
nostro caso ha una dimensione pari a $619$ byte. Questo fa sì che non si ecceda la dimensione del payload del pacchetto IP, in questo
modo la fase di \lstinline|IKE_AUTH| si effettua solamente uno scambio.


\subsubsection*{Secure}
Sono presenti le stesse configurazioni tuttavia con l'impiego di una chiper suite più sicura, questa è la seguente:
\lstinline|AES_256_SHA_384_ECP384|.
%----------------------------------------------------------------------------------------
%	TESTING SECTION
%----------------------------------------------------------------------------------------
\newpage
\section{Testing}

Il nostro lavoro si concentra sullo stimare le risorse necessarie per l'instaurazione di Security Associations (SA). In ogni sottosezione:
\begin{enumerate}
    \item analizzaremo uno degli aspetti critici considerati;
    \item descriveremo come abbiamo effettutato le misurazioni;
    \item descriveremo i risultati ottenuti.
\end{enumerate}


\subsection{Analisi traffico}

\subsubsection{Misurazioni}

Per interecettare e analizzare il traffico abbiamo realizzato uno shell script. Il tool principale per la realizzazione di quest'ultimo e' \lstinline|tcpdump|, un programma per il monitoraggio delle interfaccie e la cattura dei pacchetti in entrata ed in un uscita da un calcolatore. 
La scelta si è basata sulla precisione dei timestamp che utilizza, e alla sua diffusione (e' gia presente di default in molte distro linux).
\\ 

\noindent
Lo script iterativamente \textit{instaura} ed \textit{elimina} una SA, nel mentre cattura il traffico. Tramite questo possiamo misurare i tempi necessari per le diverse fasi del protcollo, e le dimensioni dei pacchetti.
Il tool permette di specificare la suite crittografica che si vuole prendere in esame, ed il numero di tentativi da effettuare.
Al termine delle iterazioni lo script fornisce delle medie dei valori che si vogliono misurare.

\subsubsection{Risultati}

Come specificato nella fase di configurazione si nota che il numero di scambi dipende fortemente dalla suite crittografica considerata.
In particolare, il \textbf{metodo di autenticazione} e' la componente che incide maggiormente.

\noindent
Ad esempio l'autenticazione tramite \lstinline|MSCHAPv2| prevede 4 scambi, mentre l'utilizzo di certificati ne prevede 2. 
Tuttavia e' importante osservare che qualora il certificato ecceda la dimensione massima per il payload dei pacchetti ip, si ha \textbf{frammentazione}. 
Questo fenomeno si puo' osservare ad esempio con l'utilizzo di certificati generati con \lstinline|RSA 2048|.
\\

\noindent
Di seguito riportiamo i risultati ottenuti utilizzando certificati generati rispettivamente con \lstinline|RSA 2048| e con \lstinline|ECDSA 256|, data la seguente suite crittografica: \\
\lstinline|AES_128_CBC_HMAC_256_AES_128_XCBC_ECP_256|.

\begin{figure}[ht]
    \begin{minipage}{0.45\textwidth}
      \begin{lstlisting}[caption=Risultati ECDSA, language=bash]
| Average time per attempt | 0.0587 
 ----------------------------------
| Total attempts time      | 02.9371
 ----------------------------------
| Average packets          | 4 
 ----------------------------------
| Average init bytes       | 1365 
 ----------------------------------
| Average auth bytes       | 1772 
 ----------------------------------
| Average exchanged bytes  | 3137 
 ----------------------------------
| Total exchanged bytes    | 156850
      \end{lstlisting}
    \end{minipage}\hfill
    \begin{minipage}{0.45\textwidth}
      \begin{lstlisting}[caption=Risultati RSA, language=bash]
| Average time per attempt | 0.0839
 ----------------------------------
| Total attempts time      | 04.1999
 ----------------------------------
| Average packets          | 6 
 ----------------------------------
| Average init bytes       | 1365 
 ----------------------------------
| Average auth bytes       | 4184 
 ----------------------------------
| Average exchanged bytes  | 5549 
 ----------------------------------
| Total exchanged bytes    | 277450
      \end{lstlisting}
    \end{minipage}
  \end{figure}


\subsection{Utilizzo risorse CPU}

\subsubsection{Misurazioni}

Per misurare le risorse computazionali richieste da Strongswan, abbiamo utilizzato il tool \lstinline|perf|.
Questo permette di effettuare CPU profiling, esaminando diversi parametri della CPU per l'esecuzione di un certo comando.
\\
In questo caso ci siamo soffermati sul numero di cicli di clock ed istruzioni per SA instaurata.
Inoltre \lstinline|perf| permette di misurare \textit{n} esecuzioni del comando, fornendo media e varianza dei valori.


\subsubsection{Risultati}

Di seguito riportiamo l'esecuzione del comando perf per l'instaurazione di SA che utilizzano rispettivamente \lstinline|RSA| e \lstinline|ECDSA| considerando 3 esecuzioni.
\vspace*{0.2cm}
\begin{lstlisting}
Performance counter stats for '/usr/sbin/ipsec up base-rsa' (3 runs):

    12.685.614      cycles             ( +-  6,59% )
    12.383.771      instructions       ( +-  0,61% )
    41.040.697 ns   duration_time      ( +- 58,59% )

        0,0410 +- 0,0240 seconds time elapsed  ( +- 58,59% )


\end{lstlisting}

\begin{lstlisting}
 Performance counter stats for '/usr/sbin/ipsec up base-ecdsa' (3 runs):

    12.461.099      cycles             ( +-  5,87% )
    12.146.504      instructions       ( +-  1,52% )
    25.065.990 ns   duration_time      ( +- 71,83% )

     0,0251 +- 0,0180 seconds time elapsed  ( +- 71,83% )

\end{lstlisting}

\subsection{Occupazione di memoria}

\subsubsection{Misurazioni}

Per misurare la quantita' di RAM necessaria per Strongswan abbiamo utilizzato il tool \lstinline|pmap|, che permette di avere informazioni sulla memoria allocata per un determinato processo.

\subsubsection{Risultati}

In seguito a diverse misurazioni possiamo constatare che l'utilizzo di memoria per il demone \textbf{charon} si attesta tipicamente a $10 MB$, piu in generale nell'ordine delle decine di MB, mentre per ogni SA instaurata si ha un aumento in memoria di circa $15 KB$, piu in generale nell'ordine delle decine di KiloByte. 
\vspace*{0.2cm}
\begin{lstlisting}
                 Kbytes    RSS    Dirty
---------------- ------- ------- ------- 
total kB         1128980   10692    2772

\end{lstlisting}
\newpage
\section{Conclusioni}
In seguito ai test effettuati, Strongswan si e' rivelato un'implementazione di \lstinline|IKE| sufficientemente prestante per i nostri scopi. 
Le diverse configurazioni non influiscono eccessivamente sulle risorse necessarie, sia in termini computazionali che di tempo. 
Tuttavia, dato il contesto satellitare, il numero di scambi risulta essere un fattore critico. 
Per questa ragione, suggeriamo l'utilizzo di una autenticazione basata su certificati di tipo \lstinline|ECDSA|, cosi da ridurre al minimo il tempo impiegato per la trasmissione, essendo questi di dimensione abbastanza contenuta da non essere frammentati.
\newpage
\appendix

\section{Configuration File}
\hypertarget{configuration}{}
Di seguito riportiamo i file di configurazione \lstinline|ipsec.conf| e \lstinline|ipsec.secrets| rispettivamente di 
initiator e di responder. Una possibile modifica ai file potrebbe essere quella di rendere il tutto simmetrico, allo stato 
attuale i due non possono scambiarsi di ruolo. Alcune note:

\begin{itemize}
    \item la connessione \textbf{default} definisce la configurazione comune a tutte le altre.
    \item la connessione \textbf{secure} è quella con cui specifichiamo la chiper\_suite sicura.
    \item \textbf{also} permette di realizzare l'erditarietà multipla tra le connessioni.
    \item il parametro \textbf{auto} specifica quale operazione effettuare con la connessioni all'avvio di IPsec; il valore \textit{add} la aggiunge alle possibile conessioni ma non cerca di stabilirla
\end{itemize}

\subsection{Initiator}

\subsubsection*{\lstinline|ipsec.conf|}
\begin{lstlisting}
########################################################
# ipsec.conf - strongSwan IPsec configuration file
########################################################
conn %default
    leftsourceip=%config
    right=<ip_responder>
    rightsubnet=0.0.0.0/0
    auto=add

conn secure
    ike=aes256-sha384-ecp384!

conn base-mschap
    leftauth=eap-mschapv2
    eap_identity="<identity>"
    rightauth=pubkey

conn base-rsa
    rightauth=pubkey-rsa-2048
    leftauth=pubkey-rsa-2048
    leftcert=<path_to_cert>

conn base-ecdsa
    rightauth=pubkey-ecdsa-256
    leftauth=pubkey-ecdsa-2048
    leftcert=<path_to_cert>

conn secure-rsa
    also=base-rsa
    also=secure

conn secure-ecdsa
    also=base-ecdsa
    also=secure

conn ipsec-ike
    also=secure
    also=base-mschap
\end{lstlisting}
\newpage

\subsubsection*{\lstinline|ipsec.secrets|}
\begin{lstlisting}
########################################################
# ipsec.secrets - strongSwan IPsec configuration file
########################################################
<identity> : EAP "<password>"

: ECDSA "/etc/ipsec.d/private/<key>.pem"
: RSA "/etc/ipsec.d/private/<key>.pem"

\end{lstlisting}

\subsection{Responder}

\subsubsection*{\lstinline|ipsec.conf|}

\begin{lstlisting}
########################################################
# ipsec.conf - strongSwan IPsec configuration file
########################################################
conn %default
    keyexchange=ikev2
    left=<ip_host>
    leftsubnet=0.0.0.0/0
    forceencaps=yes
    compress=no
    type=tunnel
    fragmentation=yes
    rekey=no
    right=<ip_initiator>
    rightid=%any
    rightsourceip=0.0.0.0/0
    rightdns=8.8.8.8,4.4.4.4
    auto=add

conn mschap
    rightauth=eap-mschapv2
    eap_identity=%identity
    leftcert=<path_to_cert>
    leftsendcert=always

conn rsa
    leftcert=<path_to_cert>
    leftauth=pubkey-rsa-2048
    rightauth=pubkey-rsa-2048
    
conn ecdsa
    leftcert=<path_to_cert>
    leftauth=ecdsa-256
    rightauth=ecdsa-256
    
\end{lstlisting}

\subsubsection*{\lstinline|ipsec.secrets|}

\begin{lstlisting}
<identity> : EAP "<password>"

: RSA "/etc/ipsec.d/private/<key>.pem"
: ECDSA "/etc/ipsec.d/private/<key>.pem"


\end{lstlisting}

\section{Tools}

Per instaurare la connessione IPsec si utilizza il seguente comando.
\vspace*{0.2cm}
\begin{lstlisting}
$ ipsec up <conn_name>
\end{lstlisting}
Per verificare che la SA sia stata correttamente instaurata è possibile utilizzare il seguente tool \lstinline|ip xfrm|,
il quale consente di effettuare la trasformazione dei pacchetti. Questo fornisce un interfaccia ai due database:
\begin{itemize}
    \item SAD: Security Association Database, tramite l'oggetto \lstinline|state|.
    \item SPD: Security Policy Database, tramite l'oggetto \lstinline|policy|.
\end{itemize}

\noindent
L'esecuzione del seguente comando fornisce una vista delle entry presenti nel SAD, possiamo poi utilizzare queste informazioni in wireahrk
per poter vedere il traffico tra i due host in chiaro.
\vspace*{0.2cm}
\begin{lstlisting}
$ ip xfrm state list
src <initiator> dst <responder>
    proto esp spi 0xc49d3a6d reqid 1 mode tunnel
    replay-window 0 flag af-unspec
    auth-trunc hmac(sha256) <skey> 128
    enc cbc(aes) <skey>
    anti-replay context: seq 0x0, oseq 0xc, bitmap 0x00000000
src <responder> dst <initiator>
    proto esp spi 0xca382e6d reqid 1 mode tunnel
    replay-window 32 flag af-unspec
    auth-trunc hmac(sha256) <skey> 128
    enc cbc(aes) <skey>
    anti-replay context: seq 0x0, oseq 0x0, bitmap 0x00000000
\end{lstlisting}

\subsection*{Wireshark}
Per vedere il traffico sniffato in chiaro occorre configurare il protocollo ISAKMP all'interno di wireshark, andiamo a specificare quelle che sono
le chiavi negoziate per l'autenticazione di messaggio e di cifratura.

\begin{itemize}
    \item Andare su \lstinline|Edit->Preferences->Protocols->ISAKMP|.
    \item Aggiungere all'interno della tabella le varie entry riportate tramite \lstinline|ip xfrm|
\end{itemize}

\subsection*{Shell script}
\hypertarget{misurazioni}{}
Di seguito e' riportato lo shell script per l'analisi del traffico dei pacchetti.
Non sono stati riportati gli output, lo script completo, così come tutto il materiale della tesina, sono 
contenuti all'interno della \href{https://github.com/DavideDeZuane/IKE}{\faGithub \, repository}.
Prima di eseguire lo script è vanno cambiati i permessi.

\vspace*{0.2cm}
\begin{lstlisting}
chmod +x ikev2-tester.sh
\end{lstlisting}
\noindent
I \textbf{flag} da specificare al momento dell'esecuzione dello script sono i seguenti:
\begin{itemize}
    \item \lstinline|-i|: per specificare l'interfaccia su cui catturare i pacchetti con tcpdump
    \item \lstinline|-s|: per specificare la suite crittografica da usare
    \item \lstinline|-n|: per specificare il numero di tentativi da effettuare
\end{itemize}

\begin{lstlisting}
#!/usr/bin/env bash

while getopts ":n:s:i:" option; do
    case $option in
        n)
      	    attempts="$OPTARG"
      		;;
        s)
            suite="$OPTARG"
            if [ $suite == 1 ];
			then
                conn="base-mschap"
                enc_alg="AES 128 CBC"
                auth_alg="HMAC 256"
                prf_alg="AES 128 XCBC"
                dh_alg="ECP 256"
                auth_method="EAP MSCHAPv2"
            elif [ $suite == 2 ]
            then
                conn="base-rsa"
                enc_alg="AES 128 CBC"
                auth_alg="HMAC 256"
                prf_alg="AES 128 XCBC"
                dh_alg="ECP 256"
                auth_method="X.509 RSA 2048 Certificate"
            elif [ $suite == 3 ]
            then
                conn="base-ecdsa"
                enc_alg="AES 128 CBC"
                auth_alg="HMAC 256"
                prf_alg="AES 128 XCBC"
                dh_alg="ECP 256"
                auth_method="X.509 ECDSA 256 Certificate"
            else
                $(echo "Suite must be 1 or 2 ...")
                exit 1
            fi
            ;;
        i)
            interface="$OPTARG"
            $(sudo timeout --preserve-status 0.5 tcpdump -ni $interface > /dev/null 2>&1 || $(echo "Supplied interface doesn't exist!" && exit))
            ;;
        *)
            echo "Usage: $0 [-n number_of_attempts] [-s suite] [-i interface]"
            exit 1
            ;;
    esac
done

if [ -z $suite ] || [ -z $suite ] || [ -z $interface ]
then
    echo "Set all flags!"
    echo "Usage: $0 [-n number_of_attempts] [-s suite] [-i interface]"
    exit 1
fi

cumulative_time=0
cumulative_packets=0
cumulative_size=0
cumulative_init_size=0
cumulative_auth_size=0
sudo ipsec down $conn > /dev/null
sleep 3

for (( att=1; att<=$attempts; att++ ))
do
    echo "Starting attempt $att"
    echo "Waiting to be sure connection is closed!"
    sleep 4
    echo "Wait is over!"
    echo "Dumping on port 500" && sudo timeout --preserve-status 5 tcpdump --immediate-mode -ni $interface -tt -l -w inittmpbuffer1 -Z root port 500 >/dev/null 2>&1 && echo "Dump on 500 over" &
    echo "Dumping on port 4500" && sudo timeout --preserve-status 5 tcpdump --immediate-mode -ni $interface -tt -l -w authtmpbuffer1 -Z root port 4500 >/dev/null && echo "Dump on 4500 over" && sleep 1 && echo "Closing SA!" && sudo ipsec down $conn > /dev/null &
    sleep 2 && echo "Establishing SA!" && sudo ipsec up $conn > /dev/null
    echo "Waiting now!"
    sleep 10
    echo "Waiting is over!"
    sudo tcpdump -tt -Z root -n -vvv -e -r inittmpbuffer1 > inittmpbuffer2 2>&1
    sudo tcpdump -tt -Z root -n -vvv -e -r authtmpbuffer1 > authtmpbuffer2 2>&1
    grep -B 1 "isakmp" authtmpbuffer2 > authtmpbuffer3
    grep -B 1 "isakmp" inittmpbuffer2 > inittmpbuffer3
    init_packets=$(grep -Eo "^[0-9]+" inittmpbuffer2 | wc -l | grep -Eo "^[0-9]+")
    auth_packets=$(grep -Eo "^[0-9]+" authtmpbuffer3 | wc -l | grep -Eo "^[0-9]+")
    init_start_time=$(grep -Eo "^[0-9]+.[0-9]+" inittmpbuffer3 | head -n 1)
    init_end_time=$(grep -Eo "^[0-9]+.[0-9]+" inittmpbuffer3 | tail -n 1)
    auth_start_time=$(grep -Eo "^[0-9]+.[0-9]+" authtmpbuffer3 | head -n 1)
    auth_end_time=$(grep -Eo "^[0-9]+.[0-9]+" authtmpbuffer3 | tail -n 1)
    init_time=$(echo "scale=6; $init_end_time - $init_start_time" | bc)
    auth_time=$(echo "scale=6; $auth_end_time - $auth_start_time" | bc)
    attempt_time=$(echo "scale=6; $auth_end_time - $init_start_time" | bc)
    attempt_packets=$(( init_packets + auth_packets ))
    cumulative_packets=$(( cumulative_packets + attempt_packets ))
    cumulative_time=$(echo "scale=6; $cumulative_time + $attempt_time" | bc)
    init_size=$(grep -Eo "length [0-9]+:" inittmpbuffer3 | grep -Eo "[0-9]+" | awk '{ SUM += $1} END { print SUM }')
    auth_size=$(grep -Eo "length [0-9]+:" authtmpbuffer3 | grep -Eo "[0-9]+" | awk '{ SUM += $1} END { print SUM }')
    attempt_size=$(( init_size + auth_size ))
    cumulative_size=$(( cumulative_size + attempt_size ))
    cumulative_init_size=$(( cumulative_init_size + init_size ))
    cumulative_auth_size=$(( cumulative_auth_size + auth_size ))
    sudo rm inittmpbuffer1 inittmpbuffer2 inittmpbuffer3 authtmpbuffer1 authtmpbuffer2 authtmpbuffer3
done

average_time=$(echo "scale=6; $cumulative_time / $attempts" | bc)
average_init_size=$(( cumulative_init_size / attempts ))
average_auth_size=$(( cumulative_auth_size / attempts ))
average_size=$(( cumulative_size / attempts ))
average_packets=$(( cumulative_packets / attempts ))


\end{lstlisting}


\subsection*{perf}

Per installare \lstinline|perf| sulle macchine debian create eseguire il seguente comando.
\begin{lstlisting}
$ sudo apt-get install linux-perf
\end{lstlisting}
\noindent
Il comando potrebbe dare errore a causa dei permessi, per risolvere queso problema basta sovrascrivere il contenuto
del file \lstinline|/proc/sys/kernel/perf_event_paranoid| con il valore $3$.

\newpage

\section{Certificati}
\hypertarget{certificati}{}
Andiamo a vedere a cosa è dovuta la dimensione dei certificati, per vedere il contenuto del certificato sotto forma di output testuale utilizzare il seguente comando.
\begin{lstlisting}
$ openssl x509 --in <cert> -text
\end{lstlisting}

\vspace*{0.2cm}
\noindent
Andiamo a vedere nello specifico il conteuto delle due tipologie di certifiati utilizzate per la sperimentazione:
\begin{itemize}
    \item ECDSA
    \item RSA
\end{itemize}
\vspace*{0.2cm}
\noindent
La differenza princiapali tra i due sta nella dimensione della chiave che nel nostro caso è di fondamentale importanza, in quanto evita la frammentazione del pacchetto.
Anche se fa uso di chiavi da $256$ bit ECDSA garantisce un livello di sicurezza pari a $2^{256}$.
\subsection{RSA Certificate}
\begin{lstlisting}
Certificate:
Data:
Version: 3 (0x2)
Serial Number: 3952640834610742420 (0x36da99e5a7ad4494)
Signature Algorithm: sha384WithRSAEncryption
Issuer: CN = <info>
Validity
    Not Before: May 29 08:42:06 2023 GMT
    Not After : May 27 08:42:06 2028 GMT
Subject: CN = <info>
Subject Public Key Info:
    Public Key Algorithm: rsaEncryption
        RSA Public-Key: (4096 bit)
        Modulus:
            00:c5:7d:50:95:2c:c3:42:32:b1:b8:1f:55:00:94:
            ---
        Exponent: 65537 (0x10001)
X509v3 extensions:
    X509v3 Authority Key Identifier: 
        keyid:99:C3:D7:54:F4:40:EC:DE:9C:7C:60:DC:ED:29:60:BF:75:B6:94:30

    X509v3 Subject Alternative Name: 
        DNS:192.168.122.145, IP Address:192.168.122.145
    X509v3 Extended Key Usage: 
        TLS Web Server Authentication, 1.3.6.1.5.5.8.2.2
Signature Algorithm: sha384WithRSAEncryption
 18:e9:7c:2b:ea:2f:2c:2b:a6:d4:bd:6c:94:63:41:29:f9:45:
 ---
\end{lstlisting}
\vspace*{0.2cm}
\noindent
Come possiamo osservare dall'output sono conenute numerose informazioni che dunque aumentano notevolmente 
la dimensione del certificato e quindi che portano alla frammentazione di quest'ultimo durante la fase di IKE AUTH.

\newpage
\subsection{ECDSA}

\begin{lstlisting}
Certificate:
Data:
    Version: 3 (0x2)
    Serial Number: 6875679331162392113 (0x5f6b51ec3b6e0631)
    Signature Algorithm: ecdsa-with-SHA256
    Issuer: CN = CA ECDSA
    Validity
        Not Before: May 29 14:17:10 2023 GMT
        Not After : May 27 14:17:10 2028 GMT
    Subject: CN = 192.168.122.171 ECDSA
    Subject Public Key Info:
        Public Key Algorithm: id-ecPublicKey
            Public-Key: (256 bit)
            pub:
                04:21:d7:c7:a0:6f:fd:13:1a:1e:f4:c6:5b:5c:88:
                5c:99:3e:bf:92:89:7c:b2:0d:44:d0:9a:c7:aa:c3:
                0b:fe:4a:75:3a:ca:7b:91:ee:1b:69:e7:4f:40:06:
                e1:27:ee:62:72:eb:f7:06:30:c6:47:ae:db:01:e4:
                36:62:12:3e:92
            ASN1 OID: prime256v1
            NIST CURVE: P-256
    X509v3 extensions:
        X509v3 Authority Key Identifier: 
            keyid:1A:12:82:AD:18:CF:85:0A:24:03:32:DC:D7:10:26:92:15:14:00:F9

        X509v3 Subject Alternative Name: 
            DNS:192.168.122.171, IP Address:192.168.122.171
        X509v3 Extended Key Usage: 
            TLS Web Server Authentication
Signature Algorithm: ecdsa-with-SHA256
     30:44:02:20:62:aa:81:67:fe:b7:2e:2f:13:f9:69:d4:6c:72:
     7e:a9:62:6a:db:7a:1b:af:35:b7:42:dc:42:fc:11:95:fa:d7:
     02:20:33:6f:7f:6b:a8:c4:c1:33:0e:04:7b:2f:99:14:85:ff:
     93:78:9c:ed:5d:84:58:61:76:d8:4d:b7:24:07:bd:b2

\end{lstlisting}
%------------------------------------------------



%------------------------------------------------


%----------------------------------------------------------------------------------------
%	BIBLIOGRAPHY
%----------------------------------------------------------------------------------------
\newpage
\renewcommand{\refname}{\spacedlowsmallcaps{References}} 

\bibliographystyle{unsrt}
\bibliography{sample.bib} 
%----------------------------------------------------------------------------------------

\end{document}