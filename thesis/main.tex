\documentclass[
10pt, % Main document font size
a4paper, % Paper type, use 'letterpaper' for US Letter paper
oneside, % One page layout (no page indentation)
%twoside, % Two page layout (page indentation for binding and different headers)
headinclude,footinclude, % Extra spacing for the header and footer
BCOR5mm, % Binding correction
]{scrartcl}

\usepackage{listings}
\usepackage{dirtree}
\usepackage{adjustbox}
\usepackage{hyperref}


\lstset{
    backgroundcolor=\color{white},
    language=bash,
    basicstyle=\ttfamily\small, % Imposta lo stile del testo
    keywordstyle=[1]{\color{blue}}, % Imposta il colore delle parole chiave
    morekeywords=[1]{pki, ipsec, ip, xfrm, conn, EAP, RSA, ECDSA, openssl, qemu, kvm}, 
    deletekeywords=[1]{type,in},
    keywordstyle=[2]{\color{green!40}}, % Imposta il colore delle parole chiave
    morekeywords=[2]{\$,}, 
    keywordstyle=[3]{\color{red}}, % Imposta il colore delle parole chiave
    morekeywords=[3]{leftsourceip, right, rightsubnet, auto, ike, leftauth, eap_identity, rightauth,leftcert,also,keyexchange,left,leftsubnet,forceencaps,compress,type,fragmentation,rekey,rightid,rightsourceip,rightdns,leftsendcert,src,dst,proto,auth,trunc, enc, anti,replay,window }, 
    deletekeywords=[3]{ipsec-ike},
    keywordstyle=[4]{\color{gray}}, % Imposta il colore delle parole chiave
    morekeywords=[4]{Certificate, Data, Version, Serial Number, Signature, Algorithm, Issuer, Validity, Subject, Public, Key, Info, Modulus, Exponent, Signature, Not, Before, After, Serial, Number, RSA, pub, X509v3, extensions, NIST, CURVE, ASN1, OID, Authority, Identifier, Alternative, Name, Extended, Usage },
    stringstyle=\color{red}, % Imposta il colore delle stringhe
    breaklines=true, % Esegue il wrapping delle linee se troppo lunghe
    frame=single, % Inserisce una cornice intorno al codice
    framesep=7pt,
}

\input{structure.tex} % Include the structure.tex file which specified the document structure and layout

\hyphenation{Fortran hy-phen-ation} % Specify custom hyphenation points in words with dashes where you would like hyphenation to occur, or alternatively, don't put any dashes in a word to stop hyphenation altogether

%----------------------------------------------------------------------------------------
%	TITLE AND AUTHOR(S)
%----------------------------------------------------------------------------------------

\title{\normalfont\spacedallcaps{IKEv2 Testing}} % The article title

%\subtitle{Subtitle} % Uncomment to display a subtitle

\author{\spacedlowsmallcaps{Davide De Zuane \& Rahmi El Mechri}} % The article author(s) - author affiliations need to be specified in the AUTHOR AFFILIATIONS block

\date{} % An optional date to appear under the author(s)

%----------------------------------------------------------------------------------------

\begin{document}

%----------------------------------------------------------------------------------------
%	HEADERS
%----------------------------------------------------------------------------------------

\renewcommand{\sectionmark}[1]{\markright{\spacedlowsmallcaps{#1}}} % The header for all pages (oneside) or for even pages (twoside)
%\renewcommand{\subsectionmark}[1]{\markright{\thesubsection~#1}} % Uncomment when using the twoside option - this modifies the header on odd pages
\lehead{\mbox{\llap{\small\thepage\kern1em\color{halfgray} \vline}\color{halfgray}\hspace{0.5em}\rightmark\hfil}} % The header style

\pagestyle{scrheadings} % Enable the headers specified in this block

%----------------------------------------------------------------------------------------
%	TABLE OF CONTENTS & LISTS OF FIGURES AND TABLES
%----------------------------------------------------------------------------------------

\maketitle % Print the title/author/date block

\newpage

\setcounter{tocdepth}{2} % Set the depth of the table of contents to show sections and subsections only

\tableofcontents % Print the table of contents

%\listoffigures % Print the list of figures

%\listoftables % Print the list of tables

%----------------------------------------------------------------------------------------
%	ABSTRACT
%----------------------------------------------------------------------------------------

\section*{Abstract} % This section will not appear in the table of contents due to the star (\section*)

\lipsum[1] % Dummy text

%----------------------------------------------------------------------------------------
%	AUTHOR AFFILIATIONS
%----------------------------------------------------------------------------------------

%----------------------------------------------------------------------------------------

\newpage % Start the article content on the second page, remove this if you have a longer abstract that goes onto the second page

%----------------------------------------------------------------------------------------
%	INTRODUCTION
%----------------------------------------------------------------------------------------

\section{Introduction}


%----------------------------------------------------------------------------------------
%	METHODS
%----------------------------------------------------------------------------------------

\newpage

\section{Setup}

Andiamo a vedere nel dettaglio l'ambiene e la configurazione che abbiamo utilizzato per realizzare i test. Per verificare le capacità di IKE abbiamo previsto:
\begin{itemize}
    \item $3$ modalità di autenticazione;
    \item $2$ chiper suite differenti da utlizzare.
\end{itemize}

\noindent
Nella fase di sperimentazione abbiamo utilizzato le seguenti convenzioni:
\begin{itemize}
    \item \textbf{Initiator}: l'host che invia la richiesta di stabilire una SA;
    \item \textbf{Responder}: l'host che risponde alle richieste.
\end{itemize}


\subsection{Environment}

Per simulare i due host della comunicazione abbiamo creato due macchine virtuali tramite l'utilizzo di qemu/kvm, questo per avere 
delle performance il più possibile simili a quelle reali. Le due macchine virtuali sono state create in modalità bridge, questo per evitare problemi con la modalità NAT.

\noindent
Le macchine virtuali utilizzato hanno le seguenti specifiche:

\begin{itemize}
    \item \textit{Processore}: 2 core (flag \lstinline|-smp|)
    \item \textit{Memoria}: 2048MB (flag \lstinline|-m|)
    \item \textit{OS}: Debian 11
    \item \textit{Network}: Bridge
\end{itemize}

\noindent
Le macchine virtuali sono state create utilizzando \lstinline|qemu/kvm| tramite i seguenti comandi è possibile creare la macchina virtuale.
\newline\newline\noindent
Per prima cosa è necessario creare un disco immagine.
\begin{lstlisting}
$ qemu-img create -f qcow2 disk.img 10G 
\end{lstlisting}
\vspace*{0.2cm}
Ora avviamo la macchina virtuale utilizzando il seguente comando.
\begin{lstlisting}
$ qemu-system-x86_64 -smp 2 -m 2G -hda disk.img -cdrom <debian_iso> \
    -net bridge,br=virbr0 -enable-kvm & disown
\end{lstlisting}

\vspace*{0.5cm}
\noindent
Un procedimento simile si applica per l'altra macchina virtuale. Se non si vuole proseguire in questo modo 
si può utilizzare l'interfaccia grafica fornita da \lstinline|virt-manager|.



\subsection{Configuration} 

I file e le directory coinvolte nel processo di configurazione sono i seguenti. Dato che una delle principali modifiche di IKEv2 
rispetto alla versione precedente è la possibilità di autenticazione tramite certificati.
\\

    \dirtree{%
        .1 /etc.
        .2 ipsec.conf.
        .2 ipsec.secrets.
        .2 ipsec.d.
        .3 cacerts.
        .3 certs.
        .3 private.
    }
    
\begin{itemize}
    \item Il file \lstinline|ipsec.conf|\footnotemark[1] specifica la maggior parte delle configurazioni e le informazioni di controllo per il sottosistema IPsec (ulteriori specifiche e sintassi sono disponibili al seguente \href{https://linux.die.net/man/5/ipsec.conf}{link}).
    \item Il file \lstinline|ipsec.secrets|\footnotemark[1] continene i segreti che poi verrranno utilizzati nella fase di autenticazione (ulteriori specifiche al seguenti \href{https://linux.die.net/man/5/ipsec.secrets}{link}).
\end{itemize}

\footnotetext[1]{Le configurazioni utilizzate si trovano in \hyperlink{configuration}{appendice}. }

\subsubsection*{Certificati}

Una delle principali novità che introduce \lstinline|IKEv2| è la possibilità di eseguire l'autenticazione tra certificati X.509. In fase di testing abbiamo preso in considerazione due tipi di certificati:

\begin{itemize}
    \item Certificati RSA
    \item Certificati ECDSA
\end{itemize}

\noindent
A partire da una chiave pubblica è necessario realizzare un certificato di chiave pubblica e questo richiede la chiave privata di una CA. 
Nel nostro caso ci siamo creati dei certificati da CA e li abbiamo ditribuiti manualmente tra i due host. \\

\noindent
Per la generazione abbiamo utilizzato il tool \lstinline|pki|

\subsubsection*{CA Certificate}
Partiamo con la generazione dei certificati da Certification Authority, di seguito sono riportati i due comandi da utilizzare. Ne occorrono due poichè per firmare
i certificati ECDSA occorre una chiave con lo stesso schema.
\vspace*{0.2cm}
\begin{lstlisting}
$ pki --gen --type rsa --size 2048 --outform pem > 'ca.rsa.key.pem'
$ pki --gen --type ecdsa --size 256 --outform pem > 'ca.ecdsa.key.pem'
\end{lstlisting} 

\vspace*{0.2cm}
\noindent
Ora utilizziamo la chiave privata per firmare il certificato di chiave pubblica.
\begin{lstlisting}
$ pki --self --ca --lifetime 3650 --in 'ca.<type>.key.pem' --type <type> \
     --dn "CN=CA" --outform pem > ca.<type>.cert.pem
\end{lstlisting}

\noindent
Occorre poi distribuire questi due certificati ai due host, vanno messi all'intenro della directory \lstinline|cacerts|.

\subsubsection*{Host Certificate}
Passiamo ora a generare i certificati che gli host andranno ad utilizzare nella fase di autenticazione, occorre generare la coppia chiave privata, chiave pubblica.
\vspace*{0.2cm}
\begin{lstlisting}
$ pki --gen --type ecdsa --size 256 --outform pem > 'host.ecdsa.key'
$ pki --gen --type rsa --size 2048 --outform pem > 'host.rsa.key'
\end{lstlisting} 

\vspace*{0.2cm}
\noindent
E' buona norma salvare le chiavi all'interno della directory \lstinline|private|. Ora andiamo ad estrarre la chiave pubblica 
da quella appena genrata e la firmiamo con la chiave delle CA del passo precedente.
\begin{lstlisting}
$ pki --pub --in 'host.rsa.key' --type rsa | pki --issue --lifetime 1825 \
    --cacert 'ca.rsa.cert.pem' --cakey 'ca.rsa.key.pem'                  \
    --dn "CN=<Host_IP>" --san @<Host_IP> --san <Host_IP>                 \
    -- flag serverAuth --outform pem > 'host.rsa.cert.pem'
\end{lstlisting}

\noindent
Si procede in maniera analoga con le opportune modifiche anche per il certificato ECDSA. Questi vanno poi posiizonati all'intenro della
directory \lstinline|certs|.


\subsubsection{Mschap}
Il riassunto della configurazione è mostrato in tabella, per l'initiator e il responder sono riportate le modalità della loro autenticazione.
\begin{center}
    \setlength{\arrayrulewidth}{0.4mm}
    \renewcommand{\arraystretch}{1.3}
    \begin{tabular}{|l|l|}
        \hline
        \multicolumn{2}{|c|}{\textbf{Configuration}} \\
        \hline
        \textit{Initiator} & EAP-Mschapv2 \\
        \textit{Responder} & RSA Certificate $2048$ \\
        \textit{Chiper Suite} & $AES\_CBC\_128\_HMAC\_SHA2\_256\_128\_DH\_ECP\_256$ \\
        \hline
    \end{tabular}
\end{center}

\noindent
Esaminando gli scambi di IKE AUTH osserviamo che questa modalità richiede in totale $4$ exchange. 

\subsubsection{RSA}

\begin{center}
    \setlength{\arrayrulewidth}{0.4mm}
    \renewcommand{\arraystretch}{1.3}
    \begin{tabular}{|l|l|}
        \hline
        \multicolumn{2}{|c|}{\textbf{Configuration}} \\
        \hline
        \textit{Initiator} & RSA Certificate $2048$ \\
        \textit{Responder} & RSA Certificate $2048$ \\
        \textit{Chiper Suite} & $AES\_CBC\_128\_HMAC\_SHA2\_256\_128\_DH\_ECP\_256$ \\
        \hline
    \end{tabular}
\end{center}
\vspace*{0.2cm}
\noindent
Utilizzando certificati RSA si osserva che la dimensione di un certificato eccede la dimensione massima di un pacchetto IP
per tali motivi si ha la frammentazione: ovvero il contenuto, poichè eccede la dimenisone massima del campo \textit{data} viene
spezzato in più paccetti.
\newline  \\
\noindent
Anando ad esaminare il certificato, si osserva che ha una dimensione pari a $1032$ byte, di cui abbiamo:
\begin{itemize}
    \item $256$ byte per la rappresentazoine del modulo;
    \item $1$ byte per la rappresentazione dell'esponente di cifratura
    \item $384$ byte per la firma 
    \item i restanti byte sono esaminati in  \hyperlink{certificati}{appendice}.
\end{itemize} 

\noindent
Idealmente gli scambi durante IKE AUTH dovrebbero essere $2$ ovvero i due si scambiano reciprocamente i certificati. Tuttavia,
data la dimensioni di quest'ultimi, gli scambi effettivi risultano essere in totale $4$. 


\subsubsection{ECDSA}

\begin{center}
    \setlength{\arrayrulewidth}{0.4mm}
    \renewcommand{\arraystretch}{1.3}
    \begin{tabular}{|l|l|}
        \hline
        \multicolumn{2}{|c|}{\textbf{Configuration}} \\
        \hline
        \textit{Initiator} & ECDSA Certificate 256\\
        \textit{Responder} & ECDSA Certificate 256 \\
        \textit{Chiper Suite} & $AES\_CBC\_128\_HMAC\_SHA2\_256\_128\_DH\_ECP\_256$ \\
        \hline
    \end{tabular}
\end{center}
\vspace*{0.2cm}
\noindent
Si osserva che i certificati ECDSA hanno una dimensione ridotta rispetto a quella dei certifcati RSA, infatti quello utilizzato nel
nostro caso ha una dimensione pari a $619$ byte. Questo fa sì che non si ecceda la dimensione del payload del pacchetto IP, in questo
modo la fase di IKE AUTH effettua solamente uno scambio.

%----------------------------------------------------------------------------------------
%	RESULTS AND DISCUSSION
%----------------------------------------------------------------------------------------

\newpage


\section{Testing}

\subsection{Panoramica}

Il nostro lavoro si concentra sullo stimare le risorse necessarie per l'installazione di Security Associations. In ogni sottosezione analizzeremo uno degli aspetti critici, e descriveremo come abbiamo effettuate le misurazioni ed i risultati ottenuti.
Vedi \hyperlink{}{appendice} per la configurazione strumenti di misura.

\subsection{Analisi traffico}

\subsubsection{Misurazioni}

Per effettuare l'analisi del traffico necessario per l'instaurazione di una SA tra i due nodi abbiao realizzato uno shell script (vedi Appendice).
Il tool principale per la realizzazione di quest'ultimo e' tcpdump, un programma per il monitoraggio delle interfaccie e la cattura dei pacchetti in entrata ed in un uscita da un calcolatore. La scelta e' dovuta alla precisione in termini di timestamp fornita, e alla sua diffusione (e' gia presente di default in molte distro linux).
Lo script iterativamente instaura ed elimina una security association, e ne cattura il traffico, tramite il quale possiamo misurare i tempi necessari per le diverse fasi del protcollo, e le dimensioni dei pacchetti.
Il tool permette di specificare la suite crittografica che si vuole prendere in esame, ed il numero di tentativi da effettuare.
Al termine delle iterazioni lo script fornisce delle medie dei valori che si vogliono misurare.

\subsubsection{Risultati}

Come specificato nella fase di configurazione si nota che il numero di scambi dipende fortemente dalla suite crittografica considerata.
In particolare, il metodo di autenticazione e' la componente che incide maggiormente. Ad esempio l'autenticazione tramite MSCHAPv2 prevede 4 scambi, mentre l'utilizzo di certificati prevede 2 scambi. Tuttavia e' importante osservare che qualora il certificato ecceda la dimensione massima per il payload dei pacchetti ip, si ha frammentazione. Questo fenomeno si puo' osservare ad esempio con l'utilizzo di certificati generati con RSA 2048.

In figura a e b troviamo i risultati ottenuti utilizzando certificati generati rispettivamente con RSA 2048 e con ECDSA 256, data la seguente suite crittografica:

- AES 128 CBC
- HMAC 256
- AES 128 XCBC
- ECP 256


\subsection{Utilizzo risorse CPU}

\subsubsection{Misurazioni}

Per misurare le risorse richieste dalla CPU per instaurare una SA utilizzando Strongswan, abbiamo utilizzato il tool perf.
Questo permette di effettuare CPU profiling, esaminando diversi parametri della CPU per l'esecuzione di un certo comando.
In questo caso ci siamo soffermati sul numero di cicli di clock ed istruzioni per SA instaurata.
Perf permette di eseguire il comando un certo numero di volte, fornendoci media e varianza dei valori.


\subsubsection{Risultati}

In figura possiamo vedere i dati ottenuti con perf. 

\subsection{Occupazione di memoria}

\subsubsection{Misurazioni}

Per misurare la quantita' di RAM necessaria per Strongswan abbiamo utilizzato il tool pmap, che permette di avere informazioni sulla memoria allocata per un determinato processo.

\subsubsection{Risultati}

In seguito a diverse misurazioni possiamo constatare che l'utilizzo di memoria per il demone charon si attesta tipicamente a 10 MB, piu in generale nell'ordine delle decine di MB, mentre per ogni SA instaurata si ha un aumento in memoria di circa 15 KB, piu in generale nell'ordine delle decine di KiloByte. 


\section{Conclusioni}

\newpage

\appendix

\section{Configuration File}
\hypertarget{configuration}{}
Di seguito riportiamo i file di configurazione \lstinline|ipsec.conf| e \lstinline|ipsec.secrets| rispettivamente di 
initiator e di responder. Una possibile modifica ai file potrebbe essere quella di rendere il tutto simmetrico, allo stato 
attuale i due non possono scambiarsi di ruolo. Alcune note:

\begin{itemize}
    \item la connessione \textbf{default} definisce la configurazione comune a tutte le altre.
    \item la connessione \textbf{secure} è quella con cui specifichiamo la chiper\_suite sicura.
    \item \textbf{also} permette di realizzare l'erditarietà multipla tra le connessioni.
    \item il parametro \textbf{auto} specifica quale operazione effettuare con la connessioni all'avvio di IPsec; il valore \textit{add} la aggiunge alle possibile conessioni ma non cerca di stabilirla
\end{itemize}

\subsection{Initiator}

\subsubsection*{\lstinline|ipsec.conf|}
\begin{lstlisting}
########################################################
# ipsec.conf - strongSwan IPsec configuration file
########################################################
conn %default
    leftsourceip=%config
    right=<ip_responder>
    rightsubnet=0.0.0.0/0
    auto=add

conn secure
    ike=aes256-sha384-ecp384!

conn base-mschap
    leftauth=eap-mschapv2
    eap_identity="<identity>"
    rightauth=pubkey

conn base-rsa
    rightauth=pubkey-rsa-2048
    leftauth=pubkey-rsa-2048
    leftcert=<path_to_cert>

conn base-ecdsa
    rightauth=pubkey-ecdsa-256
    leftauth=pubkey-ecdsa-2048
    leftcert=<path_to_cert>

conn secure-rsa
    also=base-rsa
    also=secure

conn secure-ecdsa
    also=base-ecdsa
    also=secure

conn ipsec-ike
    also=secure
    also=base-mschap
\end{lstlisting}
\newpage

\subsubsection*{\lstinline|ipsec.secrets|}
\begin{lstlisting}
########################################################
# ipsec.secrets - strongSwan IPsec configuration file
########################################################
<identity> : EAP "<password>"

: ECDSA "/etc/ipsec.d/private/<key>.pem"
: RSA "/etc/ipsec.d/private/<key>.pem"

\end{lstlisting}

\subsection{Responder}

\subsubsection*{\lstinline|ipsec.conf|}

\begin{lstlisting}
########################################################
# ipsec.conf - strongSwan IPsec configuration file
########################################################
conn %default
    keyexchange=ikev2
    left=<ip_host>
    leftsubnet=0.0.0.0/0
    forceencaps=yes
    compress=no
    type=tunnel
    fragmentation=yes
    rekey=no
    right=<ip_initiator>
    rightid=%any
    rightsourceip=0.0.0.0/0
    rightdns=8.8.8.8,4.4.4.4
    auto=add

conn mschap
    rightauth=eap-mschapv2
    eap_identity=%identity
    leftcert=<path_to_cert>
    leftsendcert=always

conn rsa
    leftcert=<path_to_cert>
    leftauth=pubkey-rsa-2048
    rightauth=pubkey-rsa-2048
    
conn ecdsa
    leftcert=<path_to_cert>
    leftauth=ecdsa-256
    rightauth=ecdsa-256
    
\end{lstlisting}

\subsubsection*{\lstinline|ipsec.secrets|}

\begin{lstlisting}
<identity> : EAP "<password>"

: RSA "/etc/ipsec.d/private/<key>.pem"
: ECDSA "/etc/ipsec.d/private/<key>.pem"


\end{lstlisting}

\section{Tools}

Per instaurare la connessione IPsec si utilizza il seguente comando.
\vspace*{0.2cm}
\begin{lstlisting}
$ ipsec up <conn_name>
\end{lstlisting}
Per verificare che la SA sia stata correttamente instaurata è possibile utilizzare il seguente tool \lstinline|ip xfrm|,
il quale consente di effettuare la trasformazione dei pacchetti. Questo fornisce un interfaccia ai due database:
\begin{itemize}
    \item SAD: Security Association Database, tramite l'oggetto \lstinline|state|.
    \item SPD: Security Policy Database, tramite l'oggetto \lstinline|policy|.
\end{itemize}

\noindent
L'esecuzione del seguente comando fornisce una vista delle entry presenti nel SAD, possiamo poi utilizzare queste informazioni in wireahrk
per poter vedere il traffico tra i due host in chiaro.
\vspace*{0.2cm}
\begin{lstlisting}
$ ip xfrm state list
src <initiator> dst <responder>
    proto esp spi 0xc49d3a6d reqid 1 mode tunnel
    replay-window 0 flag af-unspec
    auth-trunc hmac(sha256) <skey> 128
    enc cbc(aes) <skey>
    anti-replay context: seq 0x0, oseq 0xc, bitmap 0x00000000
src <responder> dst <initiator>
    proto esp spi 0xca382e6d reqid 1 mode tunnel
    replay-window 32 flag af-unspec
    auth-trunc hmac(sha256) <skey> 128
    enc cbc(aes) <skey>
    anti-replay context: seq 0x0, oseq 0x0, bitmap 0x00000000
\end{lstlisting}

\subsection*{Wireshark}
Per vedere il traffico sniffato in chiaro occorre configurare il protocollo ISAKMP all'interno di wireshark, andiamo a specificare quelle che sono
le chiavi negoziate per l'autenticazione di messaggio e di cifratura.

\begin{itemize}
    \item Andare su \lstinline|Edit->Preferences->Protocols->ISAKMP|.
    \item Aggiungere all'interno della tabella le varie entry riportate tramite \lstinline|ip xfrm|
\end{itemize}

\subsection{Shell script}

Di seguito e' riportato lo shell script per l'analisi del traffico dei pacchetti.
\begin{lstlisting}
#!/usr/bin/env bash

while getopts ":n:s:i:" option; do
    case $option in
        n)
      	    attempts="$OPTARG"
      		;;
        s)
            suite="$OPTARG"
            if [ $suite == 1 ];
			then
                conn="base-mschap"
                enc_alg="AES 128 CBC"
                auth_alg="HMAC 256"
                prf_alg="AES 128 XCBC"
                dh_alg="ECP 256"
                auth_method="EAP MSCHAPv2"
            elif [ $suite == 2 ]
            then
                conn="base-rsa"
                enc_alg="AES 128 CBC"
                auth_alg="HMAC 256"
                prf_alg="AES 128 XCBC"
                dh_alg="ECP 256"
                auth_method="X.509 RSA 2048 Certificate"
            elif [ $suite == 3 ]
            then
                conn="base-ecdsa"
                enc_alg="AES 128 CBC"
                auth_alg="HMAC 256"
                prf_alg="AES 128 XCBC"
                dh_alg="ECP 256"
                auth_method="X.509 ECDSA 256 Certificate"
            else
                $(echo "Suite must be 1 or 2 ...")
                exit 1
            fi
            ;;
        i)
            interface="$OPTARG"
            $(sudo timeout --preserve-status 0.5 tcpdump -ni $interface > /dev/null 2>&1 || $(echo "Supplied interface doesn't exist!" && exit))
            ;;
        *)
            echo "Usage: $0 [-n number_of_attempts] [-s suite] [-i interface]"
            exit 1
            ;;
    esac
done

if [ -z $suite ] || [ -z $suite ] || [ -z $interface ]
then
    echo "Set all flags!"
    echo "Usage: $0 [-n number_of_attempts] [-s suite] [-i interface]"
    exit 1
fi

echo ""
echo "###############################################"
echo ""
echo "Number of attempts: $attempts"
echo ""
echo "###############################################"
echo ""
echo "Chosen suite: $suite"
echo " - Encryption algorithm: $enc_alg"
echo " - Authentication algorithm: $auth_alg"
echo " - Pseudo Random Function algorithm: $prf_alg"
echo " - Diffie-Hellman Group: $dh_alg"
echo " - Authentication Method: $auth_method"
echo ""
echo "###############################################"
echo ""

cumulative_time=0
cumulative_packets=0
cumulative_size=0
cumulative_init_size=0
cumulative_auth_size=0
sudo ipsec down $conn > /dev/null
sleep 3

for (( att=1; att<=$attempts; att++ ))
do
    echo "Starting attempt $att"
    echo "Waiting to be sure connection is closed!"
    sleep 4
    echo "Wait is over!"
    echo "Dumping on port 500" && sudo timeout --preserve-status 5 tcpdump --immediate-mode -ni $interface -tt -l -w inittmpbuffer1 -Z root port 500 >/dev/null 2>&1 && echo "Dump on 500 over" &
    echo "Dumping on port 4500" && sudo timeout --preserve-status 5 tcpdump --immediate-mode -ni $interface -tt -l -w authtmpbuffer1 -Z root port 4500 >/dev/null && echo "Dump on 4500 over" && sleep 1 && echo "Closing SA!" && sudo ipsec down $conn > /dev/null &
    sleep 2 && echo "Establishing SA!" && sudo ipsec up $conn > /dev/null
    echo "Waiting now!"
    sleep 10
    echo "Waiting is over!"
    sudo tcpdump -tt -Z root -n -vvv -e -r inittmpbuffer1 > inittmpbuffer2 2>&1
    sudo tcpdump -tt -Z root -n -vvv -e -r authtmpbuffer1 > authtmpbuffer2 2>&1
    grep -B 1 "isakmp" authtmpbuffer2 > authtmpbuffer3
    grep -B 1 "isakmp" inittmpbuffer2 > inittmpbuffer3
    init_packets=$(grep -Eo "^[0-9]+" inittmpbuffer2 | wc -l | grep -Eo "^[0-9]+")
    auth_packets=$(grep -Eo "^[0-9]+" authtmpbuffer3 | wc -l | grep -Eo "^[0-9]+")
    init_start_time=$(grep -Eo "^[0-9]+.[0-9]+" inittmpbuffer3 | head -n 1)
    init_end_time=$(grep -Eo "^[0-9]+.[0-9]+" inittmpbuffer3 | tail -n 1)
    auth_start_time=$(grep -Eo "^[0-9]+.[0-9]+" authtmpbuffer3 | head -n 1)
    auth_end_time=$(grep -Eo "^[0-9]+.[0-9]+" authtmpbuffer3 | tail -n 1)
    init_time=$(echo "scale=6; $init_end_time - $init_start_time" | bc)
    auth_time=$(echo "scale=6; $auth_end_time - $auth_start_time" | bc)
    attempt_time=$(echo "scale=6; $auth_end_time - $init_start_time" | bc)
    attempt_packets=$(( init_packets + auth_packets ))
    cumulative_packets=$(( cumulative_packets + attempt_packets ))
    cumulative_time=$(echo "scale=6; $cumulative_time + $attempt_time" | bc)
    init_size=$(grep -Eo "length [0-9]+:" inittmpbuffer3 | grep -Eo "[0-9]+" | awk '{ SUM += $1} END { print SUM }')
    auth_size=$(grep -Eo "length [0-9]+:" authtmpbuffer3 | grep -Eo "[0-9]+" | awk '{ SUM += $1} END { print SUM }')
    attempt_size=$(( init_size + auth_size ))
    cumulative_size=$(( cumulative_size + attempt_size ))
    cumulative_init_size=$(( cumulative_init_size + init_size ))
    cumulative_auth_size=$(( cumulative_auth_size + auth_size ))
    echo " -------------------------------------------"
    echo "| Init time       | 0$init_time  "
    echo " -------------------------------------------"
    echo "| Init packets    | $init_packets     "
    echo " -------------------------------------------"
    echo "| Exchanged bytes | $init_size   "
    echo " -------------------------------------------"
    echo "| Auth time       | 0$auth_time "
    echo " -------------------------------------------"
    echo "| Auth_packets    | $auth_packets         "
    echo " -------------------------------------------"
    echo "| Exchanged bytes | $auth_size  "
    echo " -------------------------------------------"
    echo "| Total time      | 0$attempt_time  "
    echo " -------------------------------------------"
    echo "| Total packets   | $attempt_packets "
    echo " -------------------------------------------"
    echo "| Total bytes     | $attempt_size   "
    echo " -------------------------------------------"
    echo ""
    echo "###############################################"
    echo ""
    sudo rm inittmpbuffer1 inittmpbuffer2 inittmpbuffer3 authtmpbuffer1 authtmpbuffer2 authtmpbuffer3
done

average_time=$(echo "scale=6; $cumulative_time / $attempts" | bc)
average_init_size=$(( cumulative_init_size / attempts ))
average_auth_size=$(( cumulative_auth_size / attempts ))
average_size=$(( cumulative_size / attempts ))
average_packets=$(( cumulative_packets / attempts ))


echo "All attempts executed!"
echo " -------------------------------------------"
echo "| Average time per attempt | 0$average_time "
echo " -------------------------------------------"
echo "| Total attempts time      | 0$cumulative_time "
echo " -------------------------------------------"
echo "| Average packets          | $average_packets "
echo " -------------------------------------------"
echo "| Average init bytes       | $average_init_size "
echo " -------------------------------------------"
echo "| Average auth bytes       | $average_auth_size "
echo " -------------------------------------------"
echo "| Average exchanged bytes  | $average_size "
echo " -------------------------------------------"
echo "| Total exchanged bytes    | $cumulative_size"
echo " -------------------------------------------"
echo ""
echo "Goodbye!"
echo ""
echo "###############################################"
echo "

\end{lstlisting}

Per eseguire lo script occorre renderlo eseguibile col seguente comando:
\begin{lstlisting}
chmod +x ikev2-tester.sh
\end{lstlisting}

I flag da specificare al momento dell'esecuzione dello script sono i seguenti:
- -i: per specificare l'interfaccia su cui catturare i pacchetti con tcpdump
- -s: per specificare la suite crittografica
- -n: per specificare il numero di tentativi da effettuare

Utilizzando la configurazione speficata in \hyperlink{}{}, le suite crittografiche disponibili sono le seguenti:

\subsection{perf}

Per installare perf apt-get intstall linux-perf se da problemi con workload failed è a causa dei permessi e per risolverlo basta sovrascrivere il contenuto di

$/proc/sys/kernel/perf_event_paranoid$

per fare il report di tutto l'ambiente utilizzare pef report

\subsection{pmap}

\newpage

\section{Certificati}
\hypertarget{certificati}{}
Andiamo a vedere a cosa è dovuta la dimensione dei certificati, per vedere il contenuto del certificato sotto forma di output testuale utilizzare il seguente comando.
\begin{lstlisting}
$ openssl x509 --in <cert> -text
\end{lstlisting}

\vspace*{0.2cm}
\noindent
Andiamo a vedere nello specifico il conteuto delle due tipologie di certifiati utilizzate per la sperimentazione:
\begin{itemize}
    \item ECDSA
    \item RSA
\end{itemize}
\vspace*{0.2cm}
\noindent
La differenza princiapali tra i due sta nella dimensione della chiave che nel nostro caso è di fondamentale importanza, in quanto evita la frammentazione del pacchetto.
Anche se fa uso di chiavi da $256$ bit ECDSA garantisce un livello di sicurezza pari a $2^{256}$.
\subsection{RSA Certificate}
\begin{lstlisting}
Certificate:
Data:
Version: 3 (0x2)
Serial Number: 3952640834610742420 (0x36da99e5a7ad4494)
Signature Algorithm: sha384WithRSAEncryption
Issuer: CN = <info>
Validity
    Not Before: May 29 08:42:06 2023 GMT
    Not After : May 27 08:42:06 2028 GMT
Subject: CN = <info>
Subject Public Key Info:
    Public Key Algorithm: rsaEncryption
        RSA Public-Key: (4096 bit)
        Modulus:
            00:c5:7d:50:95:2c:c3:42:32:b1:b8:1f:55:00:94:
            ---
        Exponent: 65537 (0x10001)
X509v3 extensions:
    X509v3 Authority Key Identifier: 
        keyid:99:C3:D7:54:F4:40:EC:DE:9C:7C:60:DC:ED:29:60:BF:75:B6:94:30

    X509v3 Subject Alternative Name: 
        DNS:192.168.122.145, IP Address:192.168.122.145
    X509v3 Extended Key Usage: 
        TLS Web Server Authentication, 1.3.6.1.5.5.8.2.2
Signature Algorithm: sha384WithRSAEncryption
 18:e9:7c:2b:ea:2f:2c:2b:a6:d4:bd:6c:94:63:41:29:f9:45:
 ---
\end{lstlisting}
\vspace*{0.2cm}
\noindent
Come possiamo osservare dall'output sono conenute numerose informazioni che dunque aumentano notevolmente 
la dimensione del certificato e quindi che portano alla frammentazione di quest'ultimo durante la fase di IKE AUTH.

\newpage
\subsection{ECDSA}

\begin{lstlisting}
Certificate:
Data:
    Version: 3 (0x2)
    Serial Number: 6875679331162392113 (0x5f6b51ec3b6e0631)
    Signature Algorithm: ecdsa-with-SHA256
    Issuer: CN = CA ECDSA
    Validity
        Not Before: May 29 14:17:10 2023 GMT
        Not After : May 27 14:17:10 2028 GMT
    Subject: CN = 192.168.122.171 ECDSA
    Subject Public Key Info:
        Public Key Algorithm: id-ecPublicKey
            Public-Key: (256 bit)
            pub:
                04:21:d7:c7:a0:6f:fd:13:1a:1e:f4:c6:5b:5c:88:
                5c:99:3e:bf:92:89:7c:b2:0d:44:d0:9a:c7:aa:c3:
                0b:fe:4a:75:3a:ca:7b:91:ee:1b:69:e7:4f:40:06:
                e1:27:ee:62:72:eb:f7:06:30:c6:47:ae:db:01:e4:
                36:62:12:3e:92
            ASN1 OID: prime256v1
            NIST CURVE: P-256
    X509v3 extensions:
        X509v3 Authority Key Identifier: 
            keyid:1A:12:82:AD:18:CF:85:0A:24:03:32:DC:D7:10:26:92:15:14:00:F9

        X509v3 Subject Alternative Name: 
            DNS:192.168.122.171, IP Address:192.168.122.171
        X509v3 Extended Key Usage: 
            TLS Web Server Authentication
Signature Algorithm: ecdsa-with-SHA256
     30:44:02:20:62:aa:81:67:fe:b7:2e:2f:13:f9:69:d4:6c:72:
     7e:a9:62:6a:db:7a:1b:af:35:b7:42:dc:42:fc:11:95:fa:d7:
     02:20:33:6f:7f:6b:a8:c4:c1:33:0e:04:7b:2f:99:14:85:ff:
     93:78:9c:ed:5d:84:58:61:76:d8:4d:b7:24:07:bd:b2

\end{lstlisting}
%------------------------------------------------



%------------------------------------------------


%----------------------------------------------------------------------------------------
%	BIBLIOGRAPHY
%----------------------------------------------------------------------------------------
\newpage
\renewcommand{\refname}{\spacedlowsmallcaps{References}} % For modifying the bibliography heading

\bibliographystyle{unsrt}
\bibliography{sample.bib} % The file containing the bibliography

%----------------------------------------------------------------------------------------

\end{document}