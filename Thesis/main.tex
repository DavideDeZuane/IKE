\documentclass[
10pt, % Main document font size
a4paper, % Paper type, use 'letterpaper' for US Letter paper
oneside, % One page layout (no page indentation)
%twoside, % Two page layout (page indentation for binding and different headers)
headinclude,footinclude, % Extra spacing for the header and footer
BCOR5mm, % Binding correction
]{scrartcl}

\usepackage{listings}
\usepackage{dirtree}
\usepackage{adjustbox}
\usepackage{hyperref}


\lstset{
    backgroundcolor=\color{white},
    language=bash,
    basicstyle=\ttfamily\small, % Imposta lo stile del testo
    keywordstyle=[1]{\color{blue}}, % Imposta il colore delle parole chiave
    morekeywords=[1]{pki, ipsec, ip, xfrm, conn, EAP, RSA, ECDSA, openssl, qemu, kvm}, 
    deletekeywords=[1]{type,in},
    keywordstyle=[2]{\color{green!40}}, % Imposta il colore delle parole chiave
    morekeywords=[2]{\$,}, 
    keywordstyle=[3]{\color{red}}, % Imposta il colore delle parole chiave
    morekeywords=[3]{leftsourceip, right, rightsubnet, auto, ike, leftauth, eap_identity, rightauth,leftcert,also,keyexchange,left,leftsubnet,forceencaps,compress,type,fragmentation,rekey,rightid,rightsourceip,rightdns,leftsendcert,src,dst,proto,auth,trunc, enc, anti,replay,window }, 
    deletekeywords=[3]{ipsec-ike},
    stringstyle=\color{red}, % Imposta il colore delle stringhe
    breaklines=true, % Esegue il wrapping delle linee se troppo lunghe
    frame=single, % Inserisce una cornice intorno al codice
    framesep=7pt,
}

\input{structure.tex} % Include the structure.tex file which specified the document structure and layout

\hyphenation{Fortran hy-phen-ation} % Specify custom hyphenation points in words with dashes where you would like hyphenation to occur, or alternatively, don't put any dashes in a word to stop hyphenation altogether

%----------------------------------------------------------------------------------------
%	TITLE AND AUTHOR(S)
%----------------------------------------------------------------------------------------

\title{\normalfont\spacedallcaps{IKEv2 Testing}} % The article title

%\subtitle{Subtitle} % Uncomment to display a subtitle

\author{\spacedlowsmallcaps{Davide De Zuane \& Rahmi El Mechri}} % The article author(s) - author affiliations need to be specified in the AUTHOR AFFILIATIONS block

\date{} % An optional date to appear under the author(s)

%----------------------------------------------------------------------------------------

\begin{document}

%----------------------------------------------------------------------------------------
%	HEADERS
%----------------------------------------------------------------------------------------

\renewcommand{\sectionmark}[1]{\markright{\spacedlowsmallcaps{#1}}} % The header for all pages (oneside) or for even pages (twoside)
%\renewcommand{\subsectionmark}[1]{\markright{\thesubsection~#1}} % Uncomment when using the twoside option - this modifies the header on odd pages
\lehead{\mbox{\llap{\small\thepage\kern1em\color{halfgray} \vline}\color{halfgray}\hspace{0.5em}\rightmark\hfil}} % The header style

\pagestyle{scrheadings} % Enable the headers specified in this block

%----------------------------------------------------------------------------------------
%	TABLE OF CONTENTS & LISTS OF FIGURES AND TABLES
%----------------------------------------------------------------------------------------

\maketitle % Print the title/author/date block

\newpage

\setcounter{tocdepth}{2} % Set the depth of the table of contents to show sections and subsections only

\tableofcontents % Print the table of contents

\listoffigures % Print the list of figures

\listoftables % Print the list of tables

%----------------------------------------------------------------------------------------
%	ABSTRACT
%----------------------------------------------------------------------------------------

\section*{Abstract} % This section will not appear in the table of contents due to the star (\section*)

\lipsum[1] % Dummy text

%----------------------------------------------------------------------------------------
%	AUTHOR AFFILIATIONS
%----------------------------------------------------------------------------------------

%----------------------------------------------------------------------------------------

\newpage % Start the article content on the second page, remove this if you have a longer abstract that goes onto the second page

%----------------------------------------------------------------------------------------
%	INTRODUCTION
%----------------------------------------------------------------------------------------

\section{Introduction}

L'aumento della connettività e n

La sicurezza negli ultimi anni sta diventando sempre più importante per le comunicazioni, 

c'è sempre più bisogno di realizzare collegamenti sicuri, uno strumento molto utile a questo fine sono le VPN.

Tutte le informazion su IKE sono contenute all'interno della RFC7296\cite{rfc7296}.

\subsection{IPsec}

IPsec (Internet Protocol Security) è una suite di protocolli che fornisce sicurezza alle comunicazioni Internet a livello IP. L'uso attuale più comune di IPsec è quello di fornire una rete privata virtuale (VPN).  IKE (Internet Key Exchange)
   Internet Key Exchange) è il protocollo di negoziazione e gestione delle chiavi più comunemente munemente utilizzato per fornire chiavi negoziate e aggiornate dinamicamente per IPsec.
Introduzione a quello che è ipsec e suo funzionamento

\subsection{IKE}


Importanza di IKE per effettuare lo scambio di chiavi per stabilire la SA.

Concetti su cui si basa IKE

\subsection{Strongswan}

Parlare rapidamente di quello che è
quello che è anche l'architettura di strongswan charon.

%----------------------------------------------------------------------------------------
%	METHODS
%----------------------------------------------------------------------------------------

\newpage

\section{Setup}

Andiamo a vedere nel dettaglio l'ambiene e la configurazione che abbiamo utilizzato per realizzare i test. Per verificare le capacità di IKE abbiamo previsto:
\begin{itemize}
    \item $3$ modalità di autenticazione;
    \item $2$ chiper suite differenti da utlizzare.
\end{itemize}

\noindent
Nella fase di sperimentazione abbiamo utilizzato le seguenti convenzioni:
\begin{itemize}
    \item \textbf{Initiator}: l'host che invia la richiesta di stabilire una SA;
    \item \textbf{Responder}: l'host che risponde alle richieste.
\end{itemize}


\subsection{Environment}

Per simulare i due host della comunicazione abbiamo creato due macchine virtuali tramite l'utilizzo di qemu/kvm, questo per avere 
delle performance il più possibile simili a quelle reali. Le due macchine virtuali sono state create in modalità bridge, questo per evitare problemi con la modalità NAT.

\noindent
Le macchine virtuali utilizzato hanno le seguenti specifiche:

\begin{itemize}
    \item \textit{Processore}: 2 core (flag \lstinline|-smp|)
    \item \textit{Memoria}: 2048MB (flag \lstinline|-m|)
    \item \textit{OS}: Debian 11
    \item \textit{Network}: Bridge
\end{itemize}

\noindent
Le macchine virtuali sono state create utilizzando \lstinline|qemu/kvm| tramite i seguenti comandi è possibile creare la macchina virtuale.
\newline\newline\noindent
Per prima cosa è necessario creare un disco immagine.
\begin{lstlisting}
$ qemu-img create -f qcow2 disk.img 10G 
\end{lstlisting}
\vspace*{0.2cm}
Ora avviamo la macchina virtuale utilizzando il seguente comando.
\begin{lstlisting}
$ qemu-system-x86_64 -smp 2 -m 2G -hda disk.img -cdrom <debian_iso> \
    -net bridge,br=virbr0 -enable-kvm & disown
\end{lstlisting}

\vspace*{0.5cm}
\noindent
Un procedimento simile si applica per l'altra macchina virtuale. Se non si vuole proseguire in questo modo 
si può utilizzare l'interfaccia grafica fornita da \lstinline|virt-manager|.



\subsection{Configuration} 

I file e le directory coinvolte nel processo di configurazione sono i seguenti. Dato che una delle principali modifiche di IKEv2 
rispetto alla versione precedente è la possibilità di autenticazione tramite certificati.
\\

    \dirtree{%
        .1 /etc.
        .2 ipsec.conf.
        .2 ipsec.secrets.
        .2 ipsec.d.
        .3 cacerts.
        .3 certs.
        .3 private.
    }
    
\begin{itemize}
    \item Il file \lstinline|ipsec.conf|\footnotemark[1] specifica la maggior parte delle configurazioni e le informazioni di controllo per il sottosistema IPsec (ulteriori specifiche e sintassi sono disponibili al seguente \href{https://linux.die.net/man/5/ipsec.conf}{link}).
    \item Il file \lstinline|ipsec.secrets|\footnotemark[1] continene i segreti che poi verrranno utilizzati nella fase di autenticazione (ulteriori specifiche al seguenti \href{https://linux.die.net/man/5/ipsec.secrets}{link}).
\end{itemize}

\footnotetext[1]{Le configurazioni utilizzate si trovano in \hyperlink{configuration}{appendice}. }

\subsubsection*{Certificati}

Una delle principali novità che introduce \lstinline|IKEv2| è la possibilità di eseguire l'autenticazione tra certificati X.509. In fase di testing abbiamo preso in considerazione due tipi di certificati:

\begin{itemize}
    \item Certificati RSA
    \item Certificati ECDSA
\end{itemize}

\noindent
A partire da una chiave pubblica è necessario realizzare un certificato di chiave pubblica e questo richiede la chiave privata di una CA. 
Nel nostro caso ci siamo creati dei certificati da CA e li abbiamo ditribuiti manualmente tra i due host. \\

\noindent
Per la generazione abbiamo utilizzato il tool \lstinline|pki|

\subsubsection*{CA Certificate}
Partiamo con la generazione dei certificati da Certification Authority, di seguito sono riportati i due comandi da utilizzare. Ne occorrono due poichè per firmare
i certificati ECDSA occorre una chiave con lo stesso schema.
\vspace*{0.2cm}
\begin{lstlisting}
$ pki --gen --type rsa --size 2048 --outform pem > 'ca.rsa.key.pem'
$ pki --gen --type ecdsa --size 256 --outform pem > 'ca.ecdsa.key.pem'
\end{lstlisting} 

\vspace*{0.2cm}
\noindent
Ora utilizziamo la chiave privata per firmare il certificato di chiave pubblica.
\begin{lstlisting}
$ pki --self --ca --lifetime 3650 --in 'ca.<type>.key.pem' --type <type> \
     --dn "CN=CA" --outform pem > ca.<type>.cert.pem
\end{lstlisting}

\noindent
Occorre poi distribuire questi due certificati ai due host, vanno messi all'intenro della directory \lstinline|cacerts|.

\subsubsection*{Host Certificate}
Passiamo ora a generare i certificati che gli host andranno ad utilizzare nella fase di autenticazione, occorre generare la coppia chiave privata, chiave pubblica.
\vspace*{0.2cm}
\begin{lstlisting}
$ pki --gen --type ecdsa --size 256 --outform pem > 'host.ecdsa.key'
$ pki --gen --type rsa --size 2048 --outform pem > 'host.rsa.key'
\end{lstlisting} 

\vspace*{0.2cm}
\noindent
E' buona norma salvare le chiavi all'interno della directory \lstinline|private|. Ora andiamo ad estrarre la chiave pubblica 
da quella appena genrata e la firmiamo con la chiave delle CA del passo precedente.
\begin{lstlisting}
$ pki --pub --in 'host.rsa.key' --type rsa | pki --issue --lifetime 1825 \
    --cacert 'ca.rsa.cert.pem' --cakey 'ca.rsa.key.pem'                  \
    --dn "CN=<Host_IP>" --san @<Host_IP> --san <Host_IP>                 \
    -- flag serverAuth --outform pem > 'host.rsa.cert.pem'
\end{lstlisting}

\noindent
Si procede in maniera analoga con le opportune modifiche anche per il certificato ECDSA. Questi vanno poi posiizonati all'intenro della
directory \lstinline|certs|.


\subsubsection{Mschap}
Il riassunto della configurazione è mostrato in tabella, per l'initiator e il responder sono riportate le modalità della loro autenticazione.
\begin{center}
    \setlength{\arrayrulewidth}{0.4mm}
    \renewcommand{\arraystretch}{1.3}
    \begin{tabular}{|l|l|}
        \hline
        \multicolumn{2}{|c|}{\textbf{Configuration}} \\
        \hline
        \textit{Initiator} & EAP-Mschapv2 \\
        \textit{Responder} & RSA Certificate $2048$ \\
        \textit{Chiper Suite} & $AES\_CBC\_128\_HMAC\_SHA2\_256\_128\_DH\_ECP\_256$ \\
        \hline
    \end{tabular}
\end{center}

\noindent
Esaminando gli scambi di IKE AUTH osserviamo che questa modalità richiede in totale $4$ exchange. 

\subsubsection{RSA}

\begin{center}
    \setlength{\arrayrulewidth}{0.4mm}
    \renewcommand{\arraystretch}{1.3}
    \begin{tabular}{|l|l|}
        \hline
        \multicolumn{2}{|c|}{\textbf{Configuration}} \\
        \hline
        \textit{Initiator} & RSA Certificate $2048$ \\
        \textit{Responder} & RSA Certificate $2048$ \\
        \textit{Chiper Suite} & $AES\_CBC\_128\_HMAC\_SHA2\_256\_128\_DH\_ECP\_256$ \\
        \hline
    \end{tabular}
\end{center}

\subsubsection{ECDSA}

\begin{center}
    \setlength{\arrayrulewidth}{0.4mm}
    \renewcommand{\arraystretch}{1.3}
    \begin{tabular}{|l|l|}
        \hline
        \multicolumn{2}{|c|}{\textbf{Configuration}} \\
        \hline
        \textit{Initiator} & ECDSA Certificate 256\\
        \textit{Responder} & ECDSA Certificate 256 \\
        \textit{Chiper Suite} & $AES\_CBC\_128\_HMAC\_SHA2\_256\_128\_DH\_ECP\_256$ \\
        \hline
    \end{tabular}
\end{center}
%------------------------------------------------

%----------------------------------------------------------------------------------------
%	RESULTS AND DISCUSSION
%----------------------------------------------------------------------------------------

\newpage
\section{Testing}

Per misurare i cicli macchina abbiamo utlizzato perf 

per installarlo apt-get intstall linux-perf se da problemi con workload failed è a causa dei permessi e per risolverlo basta sovrascrivere il contenuto di


$/proc/sys/kernel/perf_event_paranoid$
per fare il report di tutto l'ambiente utilizzare pef report
\subsection{Time}

\subsection{Performance}

\subsection{Results}

\section{Conslusioni}

\newpage

\appendix

\section{Configuration File}
\hypertarget{configuration}{}
Di seguito riportiamo i file di configurazione \lstinline|ipsec.conf| e \lstinline|ipsec.secrets| rispettivamente di 
initiator e di responder. Una possibile modifica ai file potrebbe essere quella di rendere il tutto simmetrico, allo stato 
attuale i due non possono scambiarsi di ruolo. Alcune note:

\begin{itemize}
    \item la connessione \textbf{default} definisce la configurazione comune a tutte le altre.
    \item la connessione \textbf{secure} è quella con cui specifichiamo la chiper\_suite sicura.
    \item \textbf{also} permette di realizzare l'erditarietà multipla tra le connessioni.
    \item il parametro \textbf{auto} specifica quale operazione effettuare con la connessioni all'avvio di IPsec; il valore \textit{add} la aggiunge alle possibile conessioni ma non cerca di stabilirla
\end{itemize}

\subsection{Initiator}

\subsubsection*{\lstinline|ipsec.conf|}
\begin{lstlisting}
########################################################
# ipsec.conf - strongSwan IPsec configuration file
########################################################
conn %default
    leftsourceip=%config
    right=<ip_responder>
    rightsubnet=0.0.0.0/0
    auto=add

conn secure
    ike=aes256-sha384-ecp384!

conn base-mschap
    leftauth=eap-mschapv2
    eap_identity="<identity>"
    rightauth=pubkey

conn base-rsa
    rightauth=pubkey-rsa-2048
    leftauth=pubkey-rsa-2048
    leftcert=<path_to_cert>

conn base-ecdsa
    rightauth=pubkey-ecdsa-256
    leftauth=pubkey-ecdsa-2048
    leftcert=<path_to_cert>

conn secure-rsa
    also=base-rsa
    also=secure

conn secure-ecdsa
    also=base-ecdsa
    also=secure

conn ipsec-ike
    also=secure
    also=base-mschap
\end{lstlisting}
\newpage

\subsubsection*{\lstinline|ipsec.secrets|}
\begin{lstlisting}
########################################################
# ipsec.secrets - strongSwan IPsec configuration file
########################################################
<identity> : EAP "<password>"

: ECDSA "/etc/ipsec.d/private/<key>.pem"
: RSA "/etc/ipsec.d/private/<key>.pem"

\end{lstlisting}

\subsection{Responder}

\subsubsection*{\lstinline|ipsec.conf|}

\begin{lstlisting}
########################################################
# ipsec.conf - strongSwan IPsec configuration file
########################################################
conn %default
    keyexchange=ikev2
    left=<ip_host>
    leftsubnet=0.0.0.0/0
    forceencaps=yes
    compress=no
    type=tunnel
    fragmentation=yes
    rekey=no
    right=<ip_initiator>
    rightid=%any
    rightsourceip=0.0.0.0/0
    rightdns=8.8.8.8,4.4.4.4
    auto=add

conn mschap
    rightauth=eap-mschapv2
    eap_identity=%identity
    leftcert=<path_to_cert>
    leftsendcert=always

conn rsa
    leftcert=<path_to_cert>
    leftauth=pubkey-rsa-2048
    rightauth=pubkey-rsa-2048
    
conn ecdsa
    leftcert=<path_to_cert>
    leftauth=ecdsa-256
    rightauth=ecdsa-256
    
\end{lstlisting}

\subsubsection*{\lstinline|ipsec.secrets|}

\begin{lstlisting}
<identity> : EAP "<password>"

: RSA "/etc/ipsec.d/private/<key>.pem"
: ECDSA "/etc/ipsec.d/private/<key>.pem"


\end{lstlisting}

\section{Tools}

Per instaurare la connessione IPsec si utilizza il seguente comando.
\vspace*{0.2cm}
\begin{lstlisting}
$ ipsec up <conn_name>
\end{lstlisting}
Per verificare che la SA sia stata correttamente instaurata è possibile utilizzare il seguente tool \lstinline|ip xfrm|,
il quale consente di effettuare la trasformazione dei pacchetti. Questo fornisce un interfaccia ai due database:
\begin{itemize}
    \item SAD: Security Association Database, tramite l'oggetto \lstinline|state|.
    \item SPD: Security Policy Database, tramite l'oggetto \lstinline|policy|.
\end{itemize}

\noindent
L'esecuzione del seguente comando fornisce una vista delle entry presenti nel SAD, possiamo poi utilizzare queste informazioni in wireahrk
per poter vedere il traffico tra i due host in chiaro.
\vspace*{0.2cm}
\begin{lstlisting}
$ ip xfrm state list
src <initiator> dst <responder>
    proto esp spi 0xc49d3a6d reqid 1 mode tunnel
    replay-window 0 flag af-unspec
    auth-trunc hmac(sha256) <skey> 128
    enc cbc(aes) <skey>
    anti-replay context: seq 0x0, oseq 0xc, bitmap 0x00000000
src <responder> dst <initiator>
    proto esp spi 0xca382e6d reqid 1 mode tunnel
    replay-window 32 flag af-unspec
    auth-trunc hmac(sha256) <skey> 128
    enc cbc(aes) <skey>
    anti-replay context: seq 0x0, oseq 0x0, bitmap 0x00000000
\end{lstlisting}

\subsection*{Wireshark}
Per vedere il traffico sniffato in chiaro occorre configurare il protocollo ISAKMP all'interno di wireshark, andiamo a specificare quelle che sono
le chiavi negoziate per l'autenticazione di messaggio e di cifratura.

\begin{itemize}
    \item Andare su \lstinline|Edit->Preferences->Protocols->ISAKMP|.
    \item Aggiungere all'interno della tabella le varie entry riportate tramite \lstinline|ip xfrm|
\end{itemize}

\section{Certificati}

Spiegare quella che è una root CA

Come funziona la catena di certificati

Spiegare quelli che sono i campi all'interno di un certificato

Cofronto tra certificato RSA e ECDSA

Per verificare le informazioni contenute all'interno di un certificato utlilizzare il seguente comando:

\begin{lstlisting}
 $ openssl x509 --in <certificate>
\end{lstlisting}
%------------------------------------------------

%----------------------------------------------------------------------------------------
%	BIBLIOGRAPHY
%----------------------------------------------------------------------------------------

\renewcommand{\refname}{\spacedlowsmallcaps{References}} % For modifying the bibliography heading

\bibliographystyle{unsrt}
\bibliography{sample.bib} % The file containing the bibliography

%----------------------------------------------------------------------------------------

\end{document}