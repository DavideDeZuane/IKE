\documentclass[
10pt, % Main document font size
a4paper, % Paper type, use 'letterpaper' for US Letter paper
oneside, % One page layout (no page indentation)
%twoside, % Two page layout (page indentation for binding and different headers)
headinclude,footinclude, % Extra spacing for the header and footer
BCOR5mm, % Binding correction
]{scrartcl}

\usepackage{listings}
\usepackage{dirtree}
\usepackage{adjustbox}
\usepackage{hyperref}

\lstset{
    backgroundcolor=\color{white},
    language=bash,
    basicstyle=\ttfamily\small, % Imposta lo stile del testo
    keywordstyle=[1]{\color{blue}}, % Imposta il colore delle parole chiave
    morekeywords=[1]{pki}, 
    deletekeywords=[1]{type},
    keywordstyle=[2]{\color{green!40}}, % Imposta il colore delle parole chiave
    morekeywords=[2]{\$}, 
    stringstyle=\color{red}, % Imposta il colore delle stringhe
    breaklines=true, % Esegue il wrapping delle linee se troppo lunghe
    frame=single, % Inserisce una cornice intorno al codice
    frameround=12,
}

%%%%%%%%%%%%%%%%%%%%%%%%%%%%%%%%%%%%%%%%%
% Arsclassica Article
% Structure Specification File
%
% This file has been downloaded from:
% http://www.LaTeXTemplates.com
%
% Original author:
% Lorenzo Pantieri (http://www.lorenzopantieri.net) with extensive modifications by:
% Vel (vel@latextemplates.com)
%
% License:
% CC BY-NC-SA 3.0 (http://creativecommons.org/licenses/by-nc-sa/3.0/)
%
%%%%%%%%%%%%%%%%%%%%%%%%%%%%%%%%%%%%%%%%%

%----------------------------------------------------------------------------------------
%	REQUIRED PACKAGES
%----------------------------------------------------------------------------------------

\usepackage[
nochapters, % Turn off chapters since this is an article        
beramono, % Use the Bera Mono font for monospaced text (\texttt)
eulermath,% Use the Euler font for mathematics
pdfspacing, % Makes use of pdftex’ letter spacing capabilities via the microtype package
dottedtoc % Dotted lines leading to the page numbers in the table of contents
]{classicthesis} % The layout is based on the Classic Thesis style

\usepackage{arsclassica} % Modifies the Classic Thesis package

\usepackage[T1]{fontenc} % Use 8-bit encoding that has 256 glyphs

\usepackage[utf8]{inputenc} % Required for including letters with accents

\usepackage{graphicx} % Required for including images
\graphicspath{{Figures/}} % Set the default folder for images

\usepackage{enumitem} % Required for manipulating the whitespace between and within lists

\usepackage{lipsum} % Used for inserting dummy 'Lorem ipsum' text into the template

\usepackage{subfig} % Required for creating figures with multiple parts (subfigures)

\usepackage{amsmath,amssymb,amsthm} % For including math equations, theorems, symbols, etc

\usepackage{varioref} % More descriptive referencing

%----------------------------------------------------------------------------------------
%	THEOREM STYLES
%---------------------------------------------------------------------------------------

\theoremstyle{definition} % Define theorem styles here based on the definition style (used for definitions and examples)
\newtheorem{definition}{Definition}

\theoremstyle{plain} % Define theorem styles here based on the plain style (used for theorems, lemmas, propositions)
\newtheorem{theorem}{Theorem}

\theoremstyle{remark} % Define theorem styles here based on the remark style (used for remarks and notes)

%----------------------------------------------------------------------------------------
%	HYPERLINKS
%---------------------------------------------------------------------------------------

\hypersetup{
%draft, % Uncomment to remove all links (useful for printing in black and white)
colorlinks=true, breaklinks=true, bookmarks=true,bookmarksnumbered,
urlcolor=webbrown, linkcolor=RoyalBlue, citecolor=webgreen, % Link colors
pdftitle={}, % PDF title
pdfauthor={\textcopyright}, % PDF Author
pdfsubject={}, % PDF Subject
pdfkeywords={}, % PDF Keywords
pdfcreator={pdfLaTeX}, % PDF Creator
pdfproducer={LaTeX with hyperref and ClassicThesis} % PDF producer
} % Include the structure.tex file which specified the document structure and layout

\hyphenation{Fortran hy-phen-ation} % Specify custom hyphenation points in words with dashes where you would like hyphenation to occur, or alternatively, don't put any dashes in a word to stop hyphenation altogether

%----------------------------------------------------------------------------------------
%	TITLE AND AUTHOR(S)
%----------------------------------------------------------------------------------------

\title{\normalfont\spacedallcaps{IKEv2 Testing}} % The article title

%\subtitle{Subtitle} % Uncomment to display a subtitle

\author{\spacedlowsmallcaps{Davide De Zuane \& Rahmi El Mechri}} % The article author(s) - author affiliations need to be specified in the AUTHOR AFFILIATIONS block

\date{} % An optional date to appear under the author(s)

%----------------------------------------------------------------------------------------

\begin{document}

%----------------------------------------------------------------------------------------
%	HEADERS
%----------------------------------------------------------------------------------------

\renewcommand{\sectionmark}[1]{\markright{\spacedlowsmallcaps{#1}}} % The header for all pages (oneside) or for even pages (twoside)
%\renewcommand{\subsectionmark}[1]{\markright{\thesubsection~#1}} % Uncomment when using the twoside option - this modifies the header on odd pages
\lehead{\mbox{\llap{\small\thepage\kern1em\color{halfgray} \vline}\color{halfgray}\hspace{0.5em}\rightmark\hfil}} % The header style

\pagestyle{scrheadings} % Enable the headers specified in this block

%----------------------------------------------------------------------------------------
%	TABLE OF CONTENTS & LISTS OF FIGURES AND TABLES
%----------------------------------------------------------------------------------------

\maketitle % Print the title/author/date block

\newpage

\setcounter{tocdepth}{2} % Set the depth of the table of contents to show sections and subsections only

\tableofcontents % Print the table of contents

\listoffigures % Print the list of figures

\listoftables % Print the list of tables

%----------------------------------------------------------------------------------------
%	ABSTRACT
%----------------------------------------------------------------------------------------

\section*{Abstract} % This section will not appear in the table of contents due to the star (\section*)

\lipsum[1] % Dummy text

%----------------------------------------------------------------------------------------
%	AUTHOR AFFILIATIONS
%----------------------------------------------------------------------------------------

%----------------------------------------------------------------------------------------

\newpage % Start the article content on the second page, remove this if you have a longer abstract that goes onto the second page

%----------------------------------------------------------------------------------------
%	INTRODUCTION
%----------------------------------------------------------------------------------------

\section{Introduction}

L'aumento della connettività e n

La sicurezza negli ultimi anni sta diventando sempre più importante per le comunicazioni, 

c'è sempre più bisogno di realizzare collegamenti sicuri, uno strumento molto utile a questo fine sono le VPN.



\subsection{IPsec}

IPsec (Internet Protocol Security) è una suite di protocolli che fornisce sicurezza alle comunicazioni Internet a livello IP. L'uso attuale più comune di IPsec è quello di fornire una rete privata virtuale (VPN).  IKE (Internet Key Exchange)
   Internet Key Exchange) è il protocollo di negoziazione e gestione delle chiavi più comunemente munemente utilizzato per fornire chiavi negoziate e aggiornate dinamicamente per IPsec.
Introduzione a quello che è ipsec e suo funzionamento

\subsection{IKE}


Importanza di IKE per effettuare lo scambio di chiavi per stabilire la SA.

Concetti su cui si basa IKE

\subsection{Strongswan}

Parlare rapidamente di quello che è
quello che è anche l'architettura di strongswan charon.

%----------------------------------------------------------------------------------------
%	METHODS
%----------------------------------------------------------------------------------------

\newpage

\section{Setup}

Andiamo a vedere nel dettaglio l'ambiene e la configurazione che abbiamo utilizzato per realizzare i test. Per verificare le capacità di IKE abbiamo previsto:
\begin{itemize}
    \item $3$ modalità di autenticazione;
    \item $2$ chiper suite differenti da utlizzare.
\end{itemize}

\noindent
Nella fase di sperimentazione abbiamo utilizzato le seguenti convenzioni:
\begin{itemize}
    \item \textbf{Initiator}: l'host che invia la richiesta di stabilire una SA;
    \item \textbf{Responder}: l'host che risponde alle richieste.
\end{itemize}


\subsection{Environment}

Per simulare i due host della comunicazione abbiamo creato due macchine virtuali tramite l'utilizzo di qemu/kvm, questo per avere 
delle performance il più possibile simili a quelle reali. Le due macchine virtuali sono state create in modalità bridge, questo per evitare problemi con la modalità NAT. In questo modo
le due macchine appartengono ala stessa rete e questo ci fa facilita la configurazione.

\noindent
Le macchine virtuali utilizzato hanno le seguenti specifiche:

\begin{itemize}
    \item \textit{Processore}: 2 core
    \item \textit{Memoria}: 2048MB
    \item \textit{OS}: Debian 11
    \item \textit{Network}: Bridge
\end{itemize}

Setup delle macchine virtuali tramite qemu cli


Dopo le macchine virtuali la lista dei vari requirements da installare


Fare un file txt contenente tutti i pacchetti da installare in modo tale da lanciare un solo apt-get install



\subsection{Configuration} 

I file e le directory coinvolte nel processo di configurazione sono i seguenti. Dato che una delle principali modifiche di IKEv2 
rispetto alla versione precedente è la possibilità di autenticazione tramite certificati andremo ad usare le directory specificate.
\\

    \dirtree{%
        .1 /etc.
        .2 ipsec.conf.
        .2 ipsec.secrets.
        .2 ipsec.d.
        .3 cacerts.
        .3 certs.
        .3 private.
    }
    
\begin{itemize}
    \item Il file \lstinline|ipsec.conf| specifica la maggior parte delle configurazioni e le informazioni di controllo per il sottosistema IPsec (ulteriori specifiche e sintassi sono disponibili al seguente \href{https://linux.die.net/man/5/ipsec.conf}{link}).
    \item Il file \lstinline|ipsec.secrets| continene i segreti che poi verrranno utilizzati nella fase di autenticazione (ulteriori specifiche al seguenti \href{https://linux.die.net/man/5/ipsec.secrets}{link}).
\end{itemize}

\subsubsection*{Certificati}

Una delle principali novità che introduce \lstinline|IKEv2| è la possibilità di eseguire l'autenticazione tra certificati X.509. In fase di testing abbiamo preso in considerazione due tipi di certificati:

\begin{itemize}
    \item Certificati RSA
    \item Certificati ECDSA
\end{itemize}

\noindent
A partire da una chiave pubblica è necessario realizzare un certificato di chiave pubblica e questo richiede la chiave privata di una CA. 
Nel nostro caso ci siamo creati dei certificati da CA e li abbiamo ditribuiti manualmente tra i due host. \\

\noindent
Per la generazione abbiamo utilizzato il tool \lstinline|pki|

\subsubsection*{CA Certificate}
Partiamo con la generazione dei certificati da Certification Authority, di seguito sono riportati i due comandi da utilizzare. Ne occorrono due poichè per firmare
i certificati ECDSA occorre una chiave con lo stesso schema.
\begin{lstlisting}
    $ pki --gen --type rsa --size 2048 --outform pem > 'ca.rsa.key.pem'
    $ pki --gen --type ecdsa --size 256 --outform pem > 'ca.ecdsa.key.pem'
\end{lstlisting} 

\noindent
Ora utilizziamo la chiave privata per firmare il certificato di chiave pubblica.
\begin{lstlisting}
    $ pki --self --ca --lifetime 3650 --in 'ca.<type>.key.pem' --type <type> \
     --dn "CN=CA" --outform pem > ca.<type>.cert.pem
\end{lstlisting}

\noindent
Occorre poi distribuire questi due certificati ai due host, vanno messi all'intenro della directory \lstinline|cacerts|.

\subsubsection*{Host Certificate}
Passiamo ora a generare i certificati che gli host andranno ad utilizzare nella fase di autenticazione, occorre generare la coppia chiave privata, chiave pubblica.
\begin{lstlisting}
    $ pki --gen --type ecdsa --size 256 --outform pem > 'host.ecdsa.key'
    $ pki --gen --type rsa --size 2048 --outform pem > 'host.rsa.key'
\end{lstlisting} 

\noindent
E' buona norma salvare le chiavi all'interno della directory \lstinline|private|. Ora andiamo ad estrarre la chiave pubblica 
da quella appena genrata e la firmiamo con la chiave delle CA del passo precedente.
\begin{lstlisting}
    $ pki --pub --in 'host.rsa.key' --type rsa | pki --issue --lifetime 1825 \
        --cacert 'ca.rsa.cert.pem' --cakey 'ca.rsa.key.pem'                  \
        --dn "CN=<Host_IP>" --san @<Host_IP> --san <Host_IP>                 \
        -- flag serverAuth --outform pem > 'host.rsa.cert.pem'
\end{lstlisting}

\noindent
Si procede in maniera analoga con le opportune modifiche anche per il certificato ECDSA. Questi vanno poi posiizonati all'intenro della
directory \lstinline|certs|.


\subsubsection{Mschap}

\subsubsection{RSA}

\subsubsection{ECDSA}

%------------------------------------------------

%----------------------------------------------------------------------------------------
%	RESULTS AND DISCUSSION
%----------------------------------------------------------------------------------------

\newpage
\section{Testing}

Per misurare i cicli macchina abbiamo utlizzato perf 

per installarlo apt-get intstall linux-perf se da problemi con workload failed è a causa dei permessi e per risolverlo basta sovrascrivere il contenuto di


$/proc/sys/kernel/perf_event_paranoid$
per fare il report di tutto l'ambiente utilizzare pef report
\subsection{Time}

\subsection{Performance}

\subsection{Results}

\section{Conslusioni}

\newpage

\appendix

\section{Certificati}

Spiegare quella che è una root CA

Come funziona la catena di certificati

Spiegare quelli che sono i campi all'interno di un certificato

Cofronto tra certificato RSA e ECDSA

Per verificare le informazioni contenute all'interno di un certificato utlilizzare il seguente comando:

\begin{lstlisting}
    openssl x509 --in <certificate>
\end{lstlisting}
%------------------------------------------------

%----------------------------------------------------------------------------------------
%	BIBLIOGRAPHY
%----------------------------------------------------------------------------------------

\renewcommand{\refname}{\spacedlowsmallcaps{References}} % For modifying the bibliography heading

\bibliographystyle{unsrt}

\bibliography{sample.bib} % The file containing the bibliography

%----------------------------------------------------------------------------------------

\end{document}